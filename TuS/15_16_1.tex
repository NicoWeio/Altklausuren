\input{header_common.tex}

\subject{TuS}
\title{Thermodynamik und Statistik}
\date{
    Wintersemester 2017/2018
}

\begin{document}

\maketitle
\thispagestyle{empty}
% \tableofcontents
\newpage


\begin{aufgabe}{Aufgabe 1: Kurzfragen}
Die Aufgaben sind qualitativer Natur und sollen das Verständnis abfragen.
Bitte antworten Sie knapp und präzise.

Wenn Sie mit einer Formel antworten, ist dies in Ordnung.
Die in der verwendeten Formel benutzten Größen müssen in jeder Teilaufgabe bekannt sein.

\subsection{a)}
Was besagt das Poincarésche Wiederkehrtheorem und warum ist es für große Systeme irrelevant?

\subsection{b)}
Zeigen Sie, dass die Informationsentropie zweier unabhängiger Ereignisse $i \in I$ und $j \in J$ aus den Ereignismengen $I$ und $J$ mit den Wahrscheinlichkeiten $p_i$ und $q_j$ additiv ist, d.h. $S_\text{tot} = S_I + S_J$ gilt.

\subsection{c)}
Was ist die latente Wärme und wie bestimmt sie $\dv{p_\text{PG}}{T}$, d.h. die Temperaturänderung des Drucks entlang einer Phasengrenze?


\subsection{d)}
Was ist der maximal mögliche Wirkungsgrad einer Arbeitsmaschine in Kontakt mit einem kalten Wärmereservoir (Temperatur $T_1$) und einem heißen Wärmereservoir (Temperatur $T_2$)?

\subsection{e)}
Was besagt das Massenwirkungsgesetz für chemische Reaktionen
(i) allgemein und
(ii) speziell für ideale Gase?

\subsection{f)}
Wie verhält sich die spezifische Wärme $C$ bei kleinen Temperaturen in einem Fermigas? (nur Proportionalitäten gefragt)

Wie verhält sie sich in einem Bosegas mit $\omega \propto k^\alpha$ in $D$ Dimensionen?

\subsection{g)}
Erklären Sie kurz, inwiefern Korrelationen mit Potenzverhalten skalenfrei sind.

\subsection{h)}
Skizzieren Sie das typische Diagramm des Renormierungsflusses einer laufenden Kopplung in der Nähe eines kontinuierlichen Phasenübergangs.
\end{aufgabe}

\begin{aufgabe}{Aufgabe 2: Barometrische Höhenformel einer anderen Welt}
Betrachten Sie $N$ Teilchen der Masse $m$, die sich im Potential
\[
    V(z) = \alpha \sqrt{z}, \quad \alpha \in \R^+
\]
befinden.
Diese befinden sich oberhalb einer Fläche $A$ in der $x$-$y$-Ebene, d.h. es gilt $z \geq 0$.

\subsection{a)}
Bestimmen Sie die Einteilchenzustandssumme $Z_1$.

\textit{Hinweis:} Substituieren Sie das Integral über $z$ geeignet, sodass Sie auf das Integral
\[
    \int_0^\infty{e^{-\sqrt{u}} \d u} = \symup{\Gamma}(3)
\]
kommen. Die Gammafunktion hat die Eigenschaft $\symup{\Gamma}(n+1) = n!$.

\subsection{b)}
Berechnen Sie mit Hilfe von $Z_1$ die mittlere Gesamtenergie $E$.

\subsection{c)}
Bestimmen Sie die Anzahl der Teilchen $N(z) \d z$ zwischen $z$ und $z + \d z$.

\end{aufgabe}

\begin{aufgabe}{Aufgabe 3: Differentiale in der Thermodynamik}
    \subsection{a)}
    Wie lautet das totale Differential der Energie $E$ in ihren natürlichen Variablen?
    Als mögliche Variablen stehen $V, \mu, T, S, p, N$ zur Auswahl.
    Drücken Sie dabei ebenfalls alle auftretenden partiellen Ableitungen der Energie durch die Ihnen bekannten thermodynamischen Größen aus.


    Nun betrachten wir ein thermodynamisches System, das durch folgende Zustandsgleichungen mit dem Parameter $a$ beschrieben wird ($N$ sei im Folgenden konstant):
    \begin{align}
        TV^{2/3} &= 8S^3 \\
        pV &= aTS \;.
    \end{align}

    \subsection{b)}
    Untersuchen Sie, für welche Wahl von $a$ sich die Zustandsgleichungen aus dem thermodynamischen Potential $E$ ergeben, d.h. wann das Differential $\d E$ (siehe Teil \textbf{a)}) vollständig ist.

    \subsection{c)}
    Berechnen Sie für diese Wahl von $a$ die Energie $E$ in ihren natürlichen Variablen aus den Zustandsgleichungen.


    Wir lösen uns nun wieder von dem speziellen thermodynamischen System und betrachten die allgemeine Wärmekapazität bei konstantem Volumen, gegeben durch folgenden Ausdruck
    \[
        C_V = -T \b{\pdvfest{V}{T}{S}} \b{\pdvfest{p}{T}{V}} \;.
    \]

    \subsection{d)}
    Zeigen Sie, dass sich die Wärmekapazität allgemein mithilfe der Maxwell-Relationen auf die zweite Ableitung der freien Energie $F$ zurückführen lässt.

\end{aufgabe}

\begin{aufgabe}{Aufgabe 4: Dichteoperatoren}
    Gegeben sei die Matrixdarstellung eines Hamiltonoperators für ein 3-Zustandssystem in seiner Eigenbasis
    \[
        H = \colvec{\varepsilon & 0 & 0 \\ 0 & -\varepsilon & 0 \\ 0 & 0 & 0}
    \]
    und die Dichteoperatoren (zur Zeit $t=0$)
    \begin{align}
        \rho_1 &= \frac{1}{3} \colvec{1 & 0 & 0 \\ 0 & 1 & 0 \\ 0 & 0 & 1} &
        \rho_2 &= \frac{1}{2} \colvec{1 & 0 & 1 \\ 0 & 0 & 0 \\ 1 & 0 & 1} \;.
    \end{align}

    \subsection{a)}
    Überprüfen Sie für jeden der Dichteoperatoren, ob er einen reinen Zustand beschreibt.

    \subsection{b)}
    Überprüfen Sie für jeden der Dichteoperatoren, ob er stationär ist.
    Welche Differentialgleichung beschreibt die Dynamik der Zustände (eine Lösung der Differentialgleichung ist nicht erforderlich).

    \subsection{c)}
    Berechnen Sie jeweils die Erwartungswerte $\a{H}_{\rho_i}$ und $\a{H^2}_{\rho_i}$ und geben Sie die Wärmekapazität $C_V$ an.

    \subsection{d)}
    Bestimmen Sie für $\rho_1$ die Informationsentropie.
\end{aufgabe}

\begin{aufgabe}{Aufgabe 5: Kreisprozess}
    Wir betrachten einen Kreisprozess eines idealen einatomigen Gases, der sich aus den folgenden vier quasistatischen Teilprozessen zusammensetzt:
    \begin{itemize}
        \item $1 \to 2$: Isentrope Kompression
        \item $2 \to 3$: Isobare Expansion
        \item $3 \to 4$: Isentrope Expansion
        \item $4 \to 1$: Isochore Druckänderung
    \end{itemize}
    bestehenden Kreisprozess, der als Arbeitssubstanz ein ideales Gas verwendet.

    \subsection{a)}
    Zeichnen und beschriften Sie das $pV$- und das zugehörige $TS$-Diagramm für diesen Kreisprozess.
    Kennzeichnen Sie im $p,V$-Diagramm für alle Teilschritte, ob Wärme oder Arbeit zu- oder abgeführt wird.
    Ist die Gesamtarbeit $\Delta$ positiv oder negativ?

    \subsection{b)}
    Geben Sie für jeden Teilprozess die Wärmemenge $\Delta Q$ an.
    Vom System aufgenommene Wärmemengen sind hierbei positiv, abgegebene Wärmemengen entsprechend negativ.

    \subsection{c)}
    Berechnen Sie der Wirkungsgrad $\eta(T_1, T_2, T_3, T_4)$ in Abhängigkeit der Temperaturen.

    \subsection{d)}
    Nutzen Sie die adiabatischen Teilprozesse aus, um die Temperaturen durch die Volumina auszudrücken.
    Geben Sie den Wirkungsgrad ausschließlich in Abhängigkeit der Volumina an.
    Vereinfachen Sie den entstehenden Ausdruck dabei so weit wie möglich.
\end{aufgabe}

\begin{aufgabe}{Aufgabe 6: Drei identische Teilchen}
    Betrachten Sie ein System aus drei identischen, nicht-wechselwirkenden Teilchen.
    Jedes der Teilchen kann genau zwei verschiedene Zustände mit den (Einteilchen-)Energien $\epsilon_1 = -\epsilon$ und $\epsilon_2 = \epsilon$ einnehmen.
    Das System ist in thermischem Kontakt mit einem Wärmebad der Temperatur $T$.

    \subsection{a)}
    Wieviele verschiedene 3-Teilchenzustände gibt es für die Fälle, dass die Teilchen
    \begin{enumerate}[label=(\roman*)]
        \item unterscheidbar
        \item Bosonen mit Spin 0
        \item Fermionen mit Spin ½
    \end{enumerate}
    sind?

    \subsection{b)}
    Berechnen Sie die kanonische Zustandssumme des Systems und seine innere Energie $U$ als Funktion von $\beta = \frac{1}{\kB T}$ für die Fälle, dass die drei Teilchen
    \begin{enumerate}[label=(\roman*)]
        \item unterscheidbar
        \item Bosonen mit Spin 0
    \end{enumerate}
    sind.

    \subsection{c)}
    Diskutieren Sie die Grenzfälle der inneren Energie für hohe und tiefe Temperaturen für beide Fälle.
    Dabei kann die innere Energie für bestimmte Grenzfälle verschwinden, d.h. $U \to 0$.
    Geben Sie dann die führende Ordnung von $\beta$ an, mit der $U$ abfällt.

\end{aufgabe}


\begin{aufgabe}{Aufgabe 7: Chemisches Potential des idealen Bosegases}
    Betrachten Sie ein ideales Gas aus spinlosen Bosonen, welches sich in einem 2-dimensionalen Kasten befindet und an ein Bad mit Temperatur $T$ und chemischen Potential $\mu$ gekoppelt ist.
    \subsection{a)}
    Die Teilchendichte $\rho = \frac{\a{N}}{V}$ sei bekannt.
    Bestimmen Sie das chemische Potential $\mu$ als Funktion der Temperatur $T$.

    \subsection{b)}
    Wie verhält sich das chemische Potential für sehr kleine Temperaturen $T$?

    \subsection{c)}
    Diskutieren Sie kurz, ob sich bei Temperaturen $T > 0$ ein Bose-Einstein-Kondensat bilden kann.
    Begründen Sie Ihre Antwort.
\end{aufgabe}

\end{document}
