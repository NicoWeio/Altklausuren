\input{header_common.tex}

\subject{TuS}
\title{Thermodynamik und Statistik}
\date{
    Wintersemester 2017/2018
}

\begin{document}

\maketitle
\thispagestyle{empty}
% \tableofcontents
\newpage


\section{Aufgabe 1: Kurzfragen}
\subsection{a)}
Welche Aussagen machen der 2. und 3. Hauptsatz über die Entropie?

Geben Sie eine Version des 2. Hauptsatzes an, die den Begriff Entropie nicht enthält.

\subsection{b)}
Ein mit einem idealen Gas gefüllter Ballon hat ein Volumen $V_\text{B}$ und der Luftdruck betrage $p_\text{L} = \SI{E5}{\newton\per\square\meter}$.
Nun wird dieser Ballon 10 Meter tief in Wasser abgesent, das sich in vollständigem thermischen Gleichgewicht mit der Luft befindet. 
Der Schweredruck des Wassers in Tiefe $h$ ist $p_\text{W} = \rho g h$ ($\rho_\text{Wasser} = \SI{E3}{\kilogram\per\cubic\meter}$ und rechnen Sie mit $g = \SI{10}{\meter\per\square\second}$).
Wie groß ist das Volumen des Ballons in 10 Meter Tiefe?

\subsection{c)}
Wie lautet die Gibbssche Phasenregel (auftretende Größen definieren)?
Wie viele Phasen können in einem zweikomponentigen System höchstens koexistieren?

\subsection{d)}
Wie lautet die Clausius-Clapeyron-Gleichung? 
Wie sind die auftretenden Größen definiert?
Was sagt die Clausius-Clapeyron-Gleichung über die Steigung der Schmelzlinie von Wasser im $p$-$T$-Diagramm aus?

\subsection{e)}
Wie groß ist im klassischen Hochtemperaturlimes die spezifische Wärme eines linearen 3-atomigen Moleküls der Form $A – A – A$?

\subsection{f)}
Zu einem Makrozustand gehören $M$ Mikrozustände.
Wie groß ist die Entropie des Makrozustands?

\subsection{g)}
Ein einzelnes Teilchen befindet sich in einem Wärmebad mit Temperatur $T$ in einem äußeren Potential $V(\vec r)$. Wie groß ist die Wahrscheinlichkeit, dass Teilchen bei $\vec r$ zu finden (bis auf einen konstanten Normierungsfaktor)?

\subsection{h)}
Für das Photonengas gilt das Plancksche Strahlungsgesetz
\[ 
u(\omega) = \frac{\hbar\omega^3}{\pi^2c^3}\frac{1}{e^{\beta\hbar\omega} - 1} \;.
\]
Leiten Sie daraus die Gesamtenergie $E = E(T,V)$ her (bis auf einen konstanten nicht zu berechnenden Zahlfaktor in Form eines Integrals).
Welche Temperaturabhängigkeit hat die spezifische Wärme?

\subsection{i)}
Was sind die natürlichen Variablen der Energie $E$ und der freien Energie $F$?
Berechnen Sie die zur Energie 
\[ 
    E = p_0 V \exp\b{\frac{2S}{2\kB N}}    
\]
gehörige freie Energie in ihren natürlichen Variablen. 
$p_0$ ist hierbei eine einheitenbehaftete Konstante.
    
    
\subsection{j)}
Wie verhält sich der Ordnungsparameter an einem diskontinuierlichen Phasenübergang? 
Geben Sie ein Beispiel an.

\subsection{k)}
Wie lautet die kanonische Dichte $\rho = \rho(q_1, \ldots q_N, p_1, \ldots, p_N)$ eines klassischen $N$-Teilchensystems mit Hamiltonfunktion $H = H(q_1, \ldots q_N, p_1, \ldots, p_N)$ und wie berechnen sich die Zustandssumme und die Entropie eines Systems?

\section{Aufgabe 2: Punktteilchen in Kasten}
% TODO: Abbildung
Gegeben sei ein Punktteilchen mit der Masse $m$ in einem zweidimensionalen Kasten $L_x \times L_y$, 
bei dem vor der Wand $y=0$ ein Kastenpotential 
\[
\mathcal{H}_\text{pot}(y) = V_0\Theta(l-y)    
\]
existiert (siehe Abbildung).

\subsection(a)
Berechnen Sie die kanonische Zustandssumme $Z_k$ und die freie Energie $F$ des Systems.

\subsection(b)
Bestimmen Sie die mittlere potentielle Energie $\a{\mathcal{H_\text{pot}}}$.
\\
Für welche Potentialstärke $V_0$ als Funktion von $L_y$ und $l$ ist die Aufenthaltswahrscheinlichkeit des Teilchens im Bereich $\mathcal{H}_\text{pot} = V_0$ (grauer Bereich)

\subsection(c)
Berechnen Sie die innere Energie $E$.
Diskutieren Sie die beiden Fälle
\begin{enumerate}
    \item $V_0 = 0$
    \item $V_0 \to \infty$
\end{enumerate}
und erläutern Sie jeweils das Verhalten des Punktteilchens.


\section{Aufgabe 3: Kreisprozess}
Betrachten Sie einen aus den \textbf{drei} Teilprozessen
\begin{itemize}
    \item $1 \to 2$: adiabatische Expansion,
    \item $2 \to 3$: isotherme Kompression,
    \item $3 \to 1$: isonbare Erwärmung
\end{itemize}
bestehenden Kreisprozess, der als Arbeitssubstanz ein ideales Gas verwendet.

\subsection{a)}
Zeichnen Sie das zugehörige $p,V$- sowie das zugehörige $T,S$-Diagramm.
Kennzeichnen Sie im $p,V$-Diagramm für alle Teilschritte, ob Wärme oder Arbeit zu- oder abgeführt wird.
Ist die Gesamtarbeit $\symup\Delta$ positiv oder negativ?

\subsection{b)}
Berechnen Sie für jeden Teilschritt die Änderung der Wärme $\symup\Delta Q_{i \to j}$ und der Arbeit $\symup\Delta W_{i \to j}$ in Abhängigkeit der Volumina und Temperaturen sowie geeigneter thermodynamischer Koeffizienten.

\subsection{c)}
Drücken Sie das Verhältnis $p_2 / p_3$ als Funktion der Volumina $V_1$ und $V_2$ aus.
Verwenden Sie hierzu neben der Zustandsgleichung auch die Adiabatengleichung des idealen Gases.

\textit{Hinweis:} Sie müssen sich die Eigenschaften sämtlicher Teilprozesse zunutze machen.

\section{Aufgabe 4: Bose-Gas mit schwacher Wechselwirkung}

\end{document}