\documentclass[
  bibliography=totoc,     % Literatur im Inhaltsverzeichnis
  captions=tableheading,  % Tabellenüberschriften
  titlepage=firstiscover, % Titelseite ist Deckblatt
]{scrartcl}

% Paket float verbessern
\usepackage{scrhack}

% Warnung, falls nochmal kompiliert werden muss
\usepackage[aux]{rerunfilecheck}

% unverzichtbare Mathe-Befehle
\usepackage{amsmath}
% viele Mathe-Symbole
\usepackage{amssymb}
% Erweiterungen für amsmath
\usepackage{mathtools}

% Fonteinstellungen
\usepackage{fontspec}
% Latin Modern Fonts werden automatisch geladen
% Alternativ zum Beispiel:
%\setromanfont{Libertinus Serif}
%\setsansfont{Libertinus Sans}
%\setmonofont{Libertinus Mono}

% Wenn man andere Schriftarten gesetzt hat,
% sollte man das Seiten-Layout neu berechnen lassen
\recalctypearea{}

% deutsche Spracheinstellungen
\usepackage[ngerman]{babel}


\usepackage[
  math-style=ISO,    % ┐
  bold-style=ISO,    % │
  sans-style=italic, % │ ISO-Standard folgen
  nabla=upright,     % │
  partial=upright,   % ┘
  warnings-off={           % ┐
    mathtools-colon,       % │ unnötige Warnungen ausschalten
    mathtools-overbracket, % │
  },                       % ┘
]{unicode-math}

% traditionelle Fonts für Mathematik
\setmathfont{Latin Modern Math}
% Alternativ zum Beispiel:
%\setmathfont{Libertinus Math}

% \setmathfont{XITS Math}[range={scr, bfscr}]
% \setmathfont{XITS Math}[range={cal, bfcal}, StylisticSet=1]

% Zahlen und Einheiten
\usepackage[
  locale=DE,                   % deutsche Einstellungen
  separate-uncertainty=true,   % immer Unsicherheit mit \pm
  per-mode=symbol-or-fraction, % / in inline math, fraction in display math
]{siunitx}

% chemische Formeln
\usepackage[
  version=4,
  math-greek=default, % ┐ mit unicode-math zusammenarbeiten
  text-greek=default, % ┘
]{mhchem}

% richtige Anführungszeichen
\usepackage[autostyle]{csquotes}

% schöne Brüche im Text
\usepackage{xfrac}

% Standardplatzierung für Floats einstellen
\usepackage{float}
\floatplacement{figure}{htbp}
\floatplacement{table}{htbp}

% Floats innerhalb einer Section halten
\usepackage[
  section, % Floats innerhalb der Section halten
  below,   % unterhalb der Section aber auf der selben Seite ist ok
]{placeins}

% Seite drehen für breite Tabellen: landscape Umgebung
\usepackage{pdflscape}

% Captions schöner machen.
\usepackage[
  labelfont=bf,        % Tabelle x: Abbildung y: ist jetzt fett
  font=small,          % Schrift etwas kleiner als Dokument
  width=0.9\textwidth, % maximale Breite einer Caption schmaler
]{caption}
% subfigure, subtable, subref
\usepackage{subcaption}

% Grafiken können eingebunden werden
\usepackage{graphicx}

% schöne Tabellen
\usepackage{booktabs}

% Verbesserungen am Schriftbild
\usepackage{microtype}

% Literaturverzeichnis
\usepackage[
  backend=biber,
]{biblatex}
% Quellendatenbank
\addbibresource{lit.bib}
% \addbibresource{programme.bib}

% Hyperlinks im Dokument
\usepackage[
  german,
  unicode,        % Unicode in PDF-Attributen erlauben
  pdfusetitle,    % Titel, Autoren und Datum als PDF-Attribute
  pdfcreator={},  % ┐ PDF-Attribute säubern
  pdfproducer={}, % ┘
]{hyperref}
% erweiterte Bookmarks im PDF
\usepackage{bookmark}

% Trennung von Wörtern mit Strichen
\usepackage[shortcuts]{extdash}

\usepackage{parskip}

\author{%
  Nicolai Weitkemper\\%
  \href{mailto:nicolai.weitkemper@tu-dortmund.de}{nicolai.weitkemper@tu-dortmund.de}%
}
\publishers{TU Dortmund – Fakultät Physik}

\global\def\colvec#1{\begin{pmatrix}#1\end{pmatrix}}
% \global\def\b#1{\left(#1\right)}
% \global\def\pdv#1{\frac{\partial}{\partial #1}}
\global\def\dv#1#2{\frac{\d #1}{\d #2}}
\global\def\pdv#1#2{\frac{\partial #1}{\partial #2}}
\global\def\pdvfest#1#2#3{\left.\frac{\partial #1}{\partial #2}\right\rvert_{#3}}
\global\def\d{\mathrm{d}}
\global\def\intfty{\int_{-\infty}^\infty} % ha ha
\global\def\half{\frac{1}{2}}
\global\def\quarter{\frac{1}{4}}

\global\def\b#1{\left(#1\right)}
\global\def\L{\mathcal{L}}
\global\def\normtwo{\frac{1}{\sqrt{2}}}


\global\def\kB{k_\text{B}}
\global\def\b#1{\left(#1\right)}
\global\def\a#1{\left\langle#1\right\rangle}

% https://tex.stackexchange.com/a/444226
\makeatletter
\renewcommand{\@seccntformat}[1]{}
\makeatother

\NewCommandCopy{\oldDelta}{\Delta}
\renewcommand{\Delta}{\mathrm{\oldDelta}}

\newenvironment{aufgabe}[1]
    {
    \section{#1}
    }
    {
    \clearpage
    }

% TODO: Nicht die neueste Version des Standard-header_common!


\subject{TuS}
\title{Thermodynamik und Statistik}
\date{
    Wintersemester 2017/2018
}

\begin{document}

\maketitle
\thispagestyle{empty}
% \tableofcontents
\newpage


\begin{aufgabe}{Aufgabe 1: Kurzfragen}
    \subsection{a)}
    Welche Aussagen machen der 2. und 3. Hauptsatz über die Entropie?

    Geben Sie eine Version des 2. Hauptsatzes an, die den Begriff Entropie nicht enthält.

    \subsection{b)}
    Ein mit einem idealen Gas gefüllter Ballon hat ein Volumen $V_\text{B}$ und der Luftdruck betrage $p_\text{L} = \SI{E5}{\newton\per\square\meter}$.
    Nun wird dieser Ballon 10 Meter tief in Wasser abgesent, das sich in vollständigem thermischen Gleichgewicht mit der Luft befindet.
    Der Schweredruck des Wassers in Tiefe $h$ ist $p_\text{W} = \rho g h$ ($\rho_\text{Wasser} = \SI{E3}{\kilogram\per\cubic\meter}$ und rechnen Sie mit $g = \SI{10}{\meter\per\square\second}$).
    Wie groß ist das Volumen des Ballons in 10 Meter Tiefe?

    \subsection{c)}
    Wie lautet die Gibbssche Phasenregel (auftretende Größen definieren)?
    Wie viele Phasen können in einem zweikomponentigen System höchstens koexistieren?

    \subsection{d)}
    Wie lautet die Clausius-Clapeyron-Gleichung?
    Wie sind die auftretenden Größen definiert?
    Was sagt die Clausius-Clapeyron-Gleichung über die Steigung der Schmelzlinie von Wasser im $p$-$T$-Diagramm aus?

    \subsection{e)}
    Wie groß ist im klassischen Hochtemperaturlimes die spezifische Wärme eines linearen 3-atomigen Moleküls der Form $A – A – A$?

    \subsection{f)}
    Zu einem Makrozustand gehören $M$ Mikrozustände.
    Wie groß ist die Entropie des Makrozustands?

    \subsection{g)}
    Ein einzelnes Teilchen befindet sich in einem Wärmebad mit Temperatur $T$ in einem äußeren Potential $V(\vec r)$. Wie groß ist die Wahrscheinlichkeit, dass Teilchen bei $\vec r$ zu finden (bis auf einen konstanten Normierungsfaktor)?

    \subsection{h)}
    Für das Photonengas gilt das Plancksche Strahlungsgesetz
    \[
    u(\omega) = \frac{\hbar\omega^3}{\pi^2c^3}\frac{1}{e^{\beta\hbar\omega} - 1} \;.
    \]
    Leiten Sie daraus die Gesamtenergie $E = E(T,V)$ her (bis auf einen konstanten nicht zu berechnenden Zahlfaktor in Form eines Integrals).
    Welche Temperaturabhängigkeit hat die spezifische Wärme?

    \subsection{i)}
    Was sind die natürlichen Variablen der Energie $E$ und der freien Energie $F$?
    Berechnen Sie die zur Energie
    \[
        E = p_0 V \exp\b{\frac{2S}{2\kB N}}
    \]
    gehörige freie Energie in ihren natürlichen Variablen.
    $p_0$ ist hierbei eine einheitenbehaftete Konstante.


    \subsection{j)}
    Wie verhält sich der Ordnungsparameter an einem diskontinuierlichen Phasenübergang?
    Geben Sie ein Beispiel an.

    \subsection{k)}
    Wie lautet die kanonische Dichte $\rho = \rho(q_1, \ldots q_N, p_1, \ldots, p_N)$ eines klassischen $N$-Teilchensystems mit Hamiltonfunktion $H = H(q_1, \ldots q_N, p_1, \ldots, p_N)$ und wie berechnen sich die Zustandssumme und die Entropie eines Systems?
\end{aufgabe}

\begin{aufgabe}{Aufgabe 2: Punktteilchen in Kasten}
    % TODO: Abbildung
    Gegeben sei ein Punktteilchen mit der Masse $m$ in einem zweidimensionalen Kasten $L_x \times L_y$,
    bei dem vor der Wand $y=0$ ein Kastenpotential
    \[
    \mathcal{H}_\text{pot}(y) = V_0\Theta(l-y)
    \]
    existiert (siehe Abbildung).

    \subsection{a)}
    Berechnen Sie die kanonische Zustandssumme $Z_k$ und die freie Energie $F$ des Systems.

    \subsection{b)}
    Bestimmen Sie die mittlere potentielle Energie $\a{\mathcal{H_\text{pot}}}$.
    \\
    Für welche Potentialstärke $V_0$ als Funktion von $L_y$ und $l$ ist die Aufenthaltswahrscheinlichkeit des Teilchens im Bereich $\mathcal{H}_\text{pot} = V_0$ (grauer Bereich)

    \subsection{c)}
    Berechnen Sie die innere Energie $E$.
    Diskutieren Sie die beiden Fälle
    \begin{enumerate}
        \item $V_0 = 0$
        \item $V_0 \to \infty$
    \end{enumerate}
    und erläutern Sie jeweils das Verhalten des Punktteilchens.
\end{aufgabe}

\begin{aufgabe}{Aufgabe 3: Kreisprozess}
    Betrachten Sie einen aus den \textbf{drei} Teilprozessen
    \begin{itemize}
        \item $1 \to 2$: adiabatische Expansion,
        \item $2 \to 3$: isotherme Kompression,
        \item $3 \to 1$: isobare Erwärmung
    \end{itemize}
    bestehenden Kreisprozess, der als Arbeitssubstanz ein ideales Gas verwendet.

    \subsection{a)}
    Zeichnen Sie das zugehörige $p,V$- sowie das zugehörige $T,S$-Diagramm.
    Kennzeichnen Sie im $p,V$-Diagramm für alle Teilschritte, ob Wärme oder Arbeit zu- oder abgeführt wird.
    Ist die Gesamtarbeit $\Delta$ positiv oder negativ?

    \subsection{b)}
    Berechnen Sie für jeden Teilschritt die Änderung der Wärme $\Delta Q_{i \to j}$ und der Arbeit $\Delta W_{i \to j}$ in Abhängigkeit der Volumina und Temperaturen sowie geeigneter thermodynamischer Koeffizienten.

    \subsection{c)}
    Drücken Sie das Verhältnis $p_2 / p_3$ als Funktion der Volumina $V_1$ und $V_2$ aus.
    Verwenden Sie hierzu neben der Zustandsgleichung auch die Adiabatengleichung des idealen Gases.

    \textit{Hinweis:} Sie müssen sich die Eigenschaften sämtlicher Teilprozesse zunutze machen.
\end{aufgabe}

\begin{aufgabe}{Aufgabe 4: Bose-Gas mit schwacher Wechselwirkung}
    Bei niedrigen Temperaturen können die Entropie $S$ und die isotherme Kompressibilität $\kappa_T$ eines schwach wechselwirkenden Bose-Gases beschrieben werden durch
    % TODO: kompiliert lokal, aber nicht auf GitHub Actions…
    % \begin{align}
    %     S(T,V) &= \frac{5}{2}aT^{3/2}V \;, \\
    %     \kappa_T &= \frac{1}{2c}V^2 \;.
    % \end{align}
    Hierbei sind $a$ und $c$ Konstanten. Im Grenzfall niedriger Temperaturen und großer Volumen verschwinde der Druck $p$.

    \subsection{a)}
    Wie lautet das totale Differential der freien Energie $F(T,V)$ bei konstanter Teilchenzahl?
    Welche Maxwell-Relation können Sie daraus herleiten?
    Tun Sie dies.

    \subsection{b)}
    Bestimmen Sie die Zustandsgleichung des Drucks $p(T,V)$ in Abhängigkeit von Temperatur und Volumen.
    Stellen Sie hierzu zunächst das vollständige Differential von $p(T,V)$ auf und integrieren Sie anschließend.

    \subsection{c)}
    Leiten Sie ausgehend von
    \[
        \kappa_T = -\frac{1}{V}\pdvfest{V}{\rho}{T}
    \]
    einen Ausdruck für die isotherme Kompressibilität $\kappa_T$ her, der die zweite Ableitung eines thermodynamischen Potentials nach einer Zustandsvariablen beinhaltet.
    \end{aufgabe}

    \begin{aufgabe}{Aufgabe 5: Dichteoperator eines Zwei-Niveau-Systems}
    Ein quantenmechanisches System besitzt zwei Energieniveaus $\epsilon_1 = 0 < \epsilon_2$.
    Es gibt keine Übergangswahrscheinlichkeit zwischen den Niveaus und das System ist in Kontakt mit einem Wärmebad der Temperatur $T$.
    Der Hamilton-Operator $\hat H$ des System lautet also in Matrix-Darstellung:
    \[
        \colvec{0 & 0 \\ 0 & \epsilon_2} \;.
    \]

    \subsection{a)}
    Geben Sie den Dichteoperator $\hat\rho$ in Matrixdarstellung an.

    \subsection{b)}
    Berechnen Sie den Dichteoperator in den Grenzfällen $T \to 0$ und $T \to \infty$.
    Handelt es sich jeweils um einen reinen Zustand?

    \subsection{c)}
    Berechnen Sie die Entropie $S$ und die innere Energie $E$.

    \subsection{d)}
    Betrachten Sie $S$ und $E$ für die Grenzfälle $T \to 0$ und $T \to \infty$.
    Interpretieren Sie die Ergebnisse dieser Grenzfälle.
\end{aufgabe}

\begin{aufgabe}{Aufgabe 6: Reales Gas}
    Die Zustandsgleichung eines realen gases wird um einen Punkt im $p,V$-Diagramm entwickelt.
    Es ergibt sich folgende näherungsweise Zustandsgleichung
    \[
        p(t,v) = p_0 - Atv - Bv^3
    \]
    mit $v \equiv V - V_0$ und $t \equiv T - T_0$. $A$ und $B$ seien positive einheitenbehaftete Konstanten, welche die richtige Dimensionalität gewährleisten.

    \subsection{a)}
    Fertigen Sie je eine Skizze von $p(t,v)$ für $t<0$, $t=0$ und $t>0$ an.

    \subsection{b)}
    Für welche $t$ tritt thermodynamische Instabilität auf?
    Begründen Sie Ihre Antwort.


    Betrachten Sie im Folgenden nur den instabilen Fall und zeichnen Sie die berechneten Größen in Ihre Skizze ein.

    \subsection{c)}
    Wie lauten die Bestimmungsgleichungen für eine Maxwellkonstruktion um thermodynamische Stabilität zu erreichen?
    Welcher Druck $p_\text{pg}$ erfüllt die Bestimmungsgleichungen?

    \subsection{d)}
    Zwischen welchen Volumina $V_+$ und $v_-$ findet der Phasenübergang statt?

    \subsection{e)}
    Berechnen Sie die Grenzen der Bereiche, in denen metastabile Zustände auftreten können.
\end{aufgabe}

\begin{aufgabe}{Aufgabe 7: Abstraktes Fermigas}
    Diese Aufgabe behandelt ein abstraktes, ideales Fermigas aus Spin-½-Teilchen mit der allgemeinen Dispersionsrelation
    \[
        \tilde\epsilon(\vec p) = \frac{1}{\alpha} |\vec p|^\gamma, \quad \gamma \geq 1 \;.
    \]

    $\alpha$ ist eine Konstante mit passender Einheit.
    Betrachten Sie das Gas in drei Dimensionen bei der Temperatur $T=0$.

    \subsection{a)}
    Berechnen Sie die Fermienergie $\varepsilon_\text{F}$ und die Fermiwellenzahl $k_\text{F}$.
    \textit{Kontrollergebnis:}
    \[
        \varepsilon_\text{F} = \frac{1}{\alpha}\b{3\pi^2\hbar^3\frac{N}{V}}^{\gamma/3}
    \]

    \subsection{b)}
    Die Einteilchenzustandsdichte ist durch
    \[
        \rho(\varepsilon) = \frac{2}{V} \sum_{\vec p}{\delta(\varepsilon - \tilde\varepsilon(\vec p))}
        = \frac{\alpha^{3/\gamma}}{\pi^2\hbar^3} \frac{\varepsilon^{(3-\gamma)/\gamma}}{\gamma}
    \]
    gegeben.
    Bestimmen Sie hiermit die Grundzustandsenergie $E_0(N, V)$.
    Wie lautet der Koeffizient $\eta$ in
    \[
        \frac{E_0}{N} = \eta\varepsilon_\text{F} \;\text{?}
    \]

    \subsection{c)}
    Berechnen Sie den inneren Druck $P_0(N, V)$ des Fermi-Gases.
    \textit{Tipp:} Sie dürfen das Kontrollergebnis \textbf{(2)} und Formel \textbf{(4)} verwenden.
    % TOOD: Referenzen ergeben so noch keinen Sinn.

\end{aufgabe}

\end{document}
