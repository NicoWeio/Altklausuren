\documentclass[
  bibliography=totoc,     % Literatur im Inhaltsverzeichnis
  captions=tableheading,  % Tabellenüberschriften
  titlepage=firstiscover, % Titelseite ist Deckblatt
]{scrartcl}

% Paket float verbessern
\usepackage{scrhack}

% Warnung, falls nochmal kompiliert werden muss
\usepackage[aux]{rerunfilecheck}

% unverzichtbare Mathe-Befehle
\usepackage{amsmath}
% viele Mathe-Symbole
\usepackage{amssymb}
% Erweiterungen für amsmath
\usepackage{mathtools}

% Fonteinstellungen
\usepackage{fontspec}
% Latin Modern Fonts werden automatisch geladen
% Alternativ zum Beispiel:
%\setromanfont{Libertinus Serif}
%\setsansfont{Libertinus Sans}
%\setmonofont{Libertinus Mono}

% Wenn man andere Schriftarten gesetzt hat,
% sollte man das Seiten-Layout neu berechnen lassen
\recalctypearea{}

% deutsche Spracheinstellungen
\usepackage[ngerman]{babel}


\usepackage[
  math-style=ISO,    % ┐
  bold-style=ISO,    % │
  sans-style=italic, % │ ISO-Standard folgen
  nabla=upright,     % │
  partial=upright,   % ┘
  warnings-off={           % ┐
    mathtools-colon,       % │ unnötige Warnungen ausschalten
    mathtools-overbracket, % │
  },                       % ┘
]{unicode-math}

% traditionelle Fonts für Mathematik
\setmathfont{Latin Modern Math}
% Alternativ zum Beispiel:
%\setmathfont{Libertinus Math}

% \setmathfont{XITS Math}[range={scr, bfscr}]
% \setmathfont{XITS Math}[range={cal, bfcal}, StylisticSet=1]

% Zahlen und Einheiten
\usepackage[
  locale=DE,                   % deutsche Einstellungen
  separate-uncertainty=true,   % immer Unsicherheit mit \pm
  per-mode=symbol-or-fraction, % / in inline math, fraction in display math
]{siunitx}

% chemische Formeln
\usepackage[
  version=4,
  math-greek=default, % ┐ mit unicode-math zusammenarbeiten
  text-greek=default, % ┘
]{mhchem}

% richtige Anführungszeichen
\usepackage[autostyle]{csquotes}

% schöne Brüche im Text
\usepackage{xfrac}

% Standardplatzierung für Floats einstellen
\usepackage{float}
\floatplacement{figure}{htbp}
\floatplacement{table}{htbp}

% Floats innerhalb einer Section halten
\usepackage[
  section, % Floats innerhalb der Section halten
  below,   % unterhalb der Section aber auf der selben Seite ist ok
]{placeins}

% Seite drehen für breite Tabellen: landscape Umgebung
\usepackage{pdflscape}

% Captions schöner machen.
\usepackage[
  labelfont=bf,        % Tabelle x: Abbildung y: ist jetzt fett
  font=small,          % Schrift etwas kleiner als Dokument
  width=0.9\textwidth, % maximale Breite einer Caption schmaler
]{caption}
% subfigure, subtable, subref
\usepackage{subcaption}

% Grafiken können eingebunden werden
\usepackage{graphicx}

% schöne Tabellen
\usepackage{booktabs}

% Verbesserungen am Schriftbild
\usepackage{microtype}

% Literaturverzeichnis
\usepackage[
  backend=biber,
]{biblatex}
% Quellendatenbank
\addbibresource{lit.bib}
% \addbibresource{programme.bib}

% Hyperlinks im Dokument
\usepackage[
  german,
  unicode,        % Unicode in PDF-Attributen erlauben
  pdfusetitle,    % Titel, Autoren und Datum als PDF-Attribute
  pdfcreator={},  % ┐ PDF-Attribute säubern
  pdfproducer={}, % ┘
]{hyperref}
% erweiterte Bookmarks im PDF
\usepackage{bookmark}

% Trennung von Wörtern mit Strichen
\usepackage[shortcuts]{extdash}

\usepackage{parskip}

\author{%
  Nicolai Weitkemper\\%
  \href{mailto:nicolai.weitkemper@tu-dortmund.de}{nicolai.weitkemper@tu-dortmund.de}%
}
\publishers{TU Dortmund – Fakultät Physik}

\global\def\colvec#1{\begin{pmatrix}#1\end{pmatrix}}
% \global\def\b#1{\left(#1\right)}
% \global\def\pdv#1{\frac{\partial}{\partial #1}}
\global\def\dv#1#2{\frac{\d #1}{\d #2}}
\global\def\pdv#1#2{\frac{\partial #1}{\partial #2}}
\global\def\pdvfest#1#2#3{\left.\frac{\partial #1}{\partial #2}\right\rvert_{#3}}
\global\def\d{\mathrm{d}}
\global\def\intfty{\int_{-\infty}^\infty} % ha ha
\global\def\half{\frac{1}{2}}
\global\def\quarter{\frac{1}{4}}

\global\def\b#1{\left(#1\right)}
\global\def\L{\mathcal{L}}
\global\def\normtwo{\frac{1}{\sqrt{2}}}


\global\def\kB{k_\text{B}}
\global\def\b#1{\left(#1\right)}
\global\def\a#1{\left\langle#1\right\rangle}

% https://tex.stackexchange.com/a/444226
\makeatletter
\renewcommand{\@seccntformat}[1]{}
\makeatother

\NewCommandCopy{\oldDelta}{\Delta}
\renewcommand{\Delta}{\mathrm{\oldDelta}}

\newenvironment{aufgabe}[1]
    {
    \section{#1}
    }
    {
    \clearpage
    }

% TODO: Nicht die neueste Version des Standard-header_common!


\subject{TuS}
\title{Thermodynamik und Statistik}
\date{
    Wintersemester 2016/2017
}

\begin{document}

\maketitle
\thispagestyle{empty}
\newpage


\begin{aufgabe}{Aufgabe 1: Kurzfragen}
    \subsection{a)}
    Betrachten Sie ein System $B$ das Sie in zwei Teile $B_1$ und $B_2$ aufteilen.
    Wenn sich die Systeme isoliert voneinander im Gleichgewicht befinden,
    wie setzt sich die Gesamtentropie $S$ des Systems $B$ aus den Teilentropien $S_1$ und $S_2$ zusammen?
    Was können Sie über die Gesamtentropie $S$ des Systems $B$ sagen,
    wenn die Teilsysteme gekoppelt werden?
    Wird sie größer, gleich oder kleiner sein als die Gesamtentropie ohne Kopplung?

    \subsection{b)}
    Wie lautet die Zustandsgleichung des idealen Gases?
    Welche Korrekturen werden für ein reales Gas benötigt?
    Geben Sie auch den physikalischen Ursprung der Korrekturterme an.


    \subsection{c)}
    Von welchen thermodynamischen Variablen hängt die freie Energie $F$ ab?
    Durch welche Transformation erhält man die innere Energie $U$ und von welchen Variablen hängt $U$ ab?
    Wie gewinnt man die Entropie $S$ und den Druck aus $F$?
    Ist $F$ eine extensive oder eine intensive Größe und warum?

    \subsection{d)}
    Wie lautet die mittlere Besetzung $n$ eines bosonischen Zustands mit der Energie $\epsilon$ bei der Temperatur $T$?
    Wann benötigen Sie für die Besetzung ein chemisches Potential?
    Wenn Sie die Dispersion des Lichts zugrunde legen,
    wie lautet dann die Energiedichte einer thermischen Lichtwelle (Plancksches Strahlungsgesetz)?

    \textit{Hinweis:} Summieren Sie alle $k$-Moden mit konstanter Energie auf und berücksichtigen Sie,
    dass das Licht zwei Polarisationsrichtungen besitzt.

    \subsection{e)}
    Wie berechnet man die kanonische Zustandssumme $Z$ für ein diskretes System?
    Geben Sie den Wert von $Z$ für ein System von $N$ unabhängigen Spins $s = \frac{1}{2}$ in einem endlichen externen Magnetfeld $B$ an.
    Wie lautet die Entropie $S$ des Systems im Grenzfall $B \to 0$?
    Woran sehen Sie, dass die Entropie extensiv ist?

    \subsection{f)}
    Wie unterscheiden sich die beiden spezifischen Wärmen $C_p$ und $C_v$?
    Wie lautet die Verknüpfung von Entropie und spezifischer Wärme?
    Aus welchen verschiedenen thermodynamischen Potentialen gewinnt man $C_v$ oder $C_p$ und warum?
\end{aufgabe}

\begin{aufgabe}{Aufgabe 2: Klassisches Gas im Potential}
    $N$ nicht miteinander wechselwirkende unterscheidbare Teilchen der Masse $m$ befinden sich im Inneren eines \textit{endlichen} Würfels mit Seitenlänge $L$ mit dem Potential
    \[
        V(\vec r) = \alpha(x + y + z), \quad \alpha > 0 \;.
    \]
    Der Würfel befindet sich im ersten Quadranten des Koordinatensystems und eine seiner Ecken berührt den Koordinatenursprung.

    \subsection{a)}
    Berechnen Sie die klassische kanonische Zustandssumme

    \subsection{b)}
    Berechnen Sie aus der kanonischen Zustandssumme die freie Energie $F$.
    Geben Sie den Zusammenhang zwischen der freien Energie $F$ und dem Druck $p$ an und berechnen Sie den Druck $p$.

    \subsection{c)}
    Wie lautet der Druck im Grenzfall $\alpha \to 0$?
    Entwickeln Sie dazu die Exponentialfunktion bis zur ersten Ordnung.
    Welche Zustandsgleichung ergibt sich?
\end{aufgabe}


\begin{aufgabe}{Aufgabe 3: Kreisprozess}
    Betrachten Sie folgenden Kreisprozess, der sich aus den vier angegebenen Teilprozessen zusammensetzt.
    Das Arbeitsmaterial ist ein ideales Gas.
    \begin{itemize}
        \item $1 \to 2$: adiabatische Kompression
        \item $2 \to 3$: isochore Druckerhöhung
        \item $3 \to 4$: adiabatische Expansion
        \item $4 \to 1$: isochore Druckverringerung
    \end{itemize}

    \subsection{a)}
    Zeichnen und beschriften Sie das $pV$- und das $TS$-Diagramm für diesen Kreisprozess.

    \textit{Hinweis:} Für adiabatische Prozesse gilt die Adiabatengleichung
    \[
        pV^\gamma = \text{const.} \;,
    \]
    wobei $\gamma = \frac{C_p}{C_V} = \frac{5}{3}$ für ein ideales Gas ist.

    \subsection{b)}
    Berechnen Sie für jeden Teilprozess die verrichtete Arbeit und Wärmeänderung.
    Sie brauchen nicht explizit angeben, ob dem Gas Wärme/Arbeit zu- oder abgeführt wird.

    \subsection{c)}
    Drücken Sie die gesamte Arbeit $W_\text{ges}(V_1, V_2, p_2, p_3)$ als Funktion der beiden Volumina sowie der beiden Drücke $p_2$ und $p_3$ aus.

    \subsection{d)}
    Berechnen Sie den Wirkungsgrad $\eta(V_1, V_2)$ als Funktion der beiden Volumina.
\end{aufgabe}

\begin{aufgabe}{Aufgabe 4: Drei-Zustand-System}
    Wir betrachten ein Teilchen in einem Zwei-Niveau-System mit den Energieniveaus $\epsilon_0$ und $\epsilon_1$ ($\epsilon_0 < \epsilon_1$).
    Der angeregte Zustand $\epsilon_1$ sei zweifach entartet und der Grundzustand $\epsilon_0$ sei nicht entartet.
    Der Hamiltonoperator enthalte keine Übergänge zwischen den beiden Niveaus.
    Das System stehe in Kontakt mit einem Wärmebad der Temperatur $T$.

    \subsection{a)}
    Geben Sie den Hamiltonoperator sowie den Dichteoperator im thermischen Gleichgewicht explizit als Matrizen an.


    Setzen Sie nun für die folgenden Aufgabenteile $\epsilon_0 = 0$ und $\epsilon_1 = \epsilon$.

    \subsection{b)}
    Beschreibt der Dichteoperator
    \begin{enumerate}[label=(\roman*)]
        \item im Grenzfall $T \to 0$
        \item im Grenzfall $T \to \infty$
    \end{enumerate}
    einen reinen Zustand?
    Interpretieren Sie Ihr Ergebnis physikalisch.

    \subsection{c)}
    Bestimmen Sie die innere Energie $U$ und die spezifische Wärme $C_V$.
    Skizzieren Sie den Verlauf der spezifischen Wärme $C_V$ in Abhängigkeit von der Temperatur $T$.
\end{aufgabe}

\begin{aufgabe}{Aufgabe 5: Thermodynamische Potentiale}
    Betrachten Sie eine Substanz, die durch die beiden Zustandsgleichungen
    \[
        pV = \alpha \b{\frac{TV^{1/3}}{\sqrt{6}}}^2
    \]
    und
    \[
        T = \b{\frac{\sqrt{\tau S}} {V^{1/3}\sqrt{\alpha}}}^2
    \]
    charakterisiert ist,
    wobei $\tau \in \R$, und $\alpha = \SI{1}{\newton\per\meter\square\kelvin}$ eine Einheiten-behaftete Konstante ist.

    \subsection{a)}
    Wie lautet das totale Differential der inneren Energie $E$ in ihren natürlichen Variablen?
    Drücken Sie dabei alle auftretenden partiellen Ableitungen der Energie durch die Ihnen bekannten thermodynamischen Größen aus.
    Leiten Sie dann daraus das vollständige Differential der freien Energie $F$ her.
    Was sind hier die natürlichen Variablen?


    Im Folgenden soll das Differential der freien Energie $F$ bei konstanter Teilchenzahl untersucht werden.

    \subsection{b)}
    Für welche Wahl von $\tau$ sind die Zustandsgleichungen sinnvoll,
    wenn sich diese aus dem vollständigen Differential der freien Energie $F$ ergeben sollen?
    \textit{Lösen Sie diese Teilaufgabe ohne zu integrieren.}

    \subsection{c)}
    Rekonstruieren Sie für diese Wahl von $\tau$ die freie Energie $F$ aus den Zustandsgleichungen.
    Achten Sie darauf, diese in den natürlichen Größen anzugeben!


    Die spezifische Wärme bei konstantem Volumen ist durch
    \[
        C_V = -T \b{\pdvfest{V}{T}{S}} \cdot \b{\pdvfest{S}{V}{T}}
    \]
    gegeben.

    \subsection{d)}
    Bestimmen Sie die spezifische Wärme als zweite Ableitung der freien Energie und berechnen Sie $C_V$.
    \textit{Geben Sie in jedem Schritt explizit die Maxwell-Relation an, die Sie benutzt haben.}
\end{aufgabe}

\begin{aufgabe}{Aufgabe 6: Ideale Quantengase}
    Im Folgenden wollen wir ideale Quantengase mit einem vereinfachten Modell beschreiben,
    in welchem die Teilchen einen einzigen Zustand der Energie $\epsilon$ besetzen können.

    \subsection{a)}
    Welche Werte kann die Besetzungszahl für Bosonen beziehungsweise Fermionen in diesem System annehmen?

    \subsection{b)}
    Berechnen Sie die großkanonische Zustandssumme $Z_\text{gk}$ und das großkanonische Potential $\Phi$ für den bosonische und den fermionischen Fall.

    \textit{Sollten Sie diesen Aufgabenteil nicht lösen können, dürfen Sie mit folgendem Ergebnis weiter rechnen:}
    \[
        \Phi = \pm k_B T \ln(1 \mp e^{\beta(\mu-\epsilon)})
    \]

    \subsection{c)}
    Berechnen Sie nun die mittlere Besetzungszahl $N(T, \epsilon)$.

    \subsection{d)}
    Welche Teilchenzahl ergibt sich im verdünnten Limes (Fugazität $z \ll 1$)?

    \subsection{e)}
    Wie heißen die Funktionen, welche Sie in Aufgabenteil \textbf{c)} und \textbf{d)} berechnet haben?
    Skizzieren Sie deren Verlauf.

    \textit{Diesen Aufgabenteil können Sie auch lösen, ohne die vorherigen Teilaufgaben explizit berechnet zu haben.}
\end{aufgabe}

\begin{aufgabe}{Aufgabe 7: Drei identische Teilchen}
    Betrachten Sie ein System aus drei identischen, nicht-wechselwirkenden Teilchen.
    Jedes der Teilchen kann genau zwei verschiedene Zustände mit den (Einteilchen-)Energien $\epsilon_1 = -\epsilon$ und $\epsilon_2 = \epsilon$ einnehmen.
    Das System ist in thermischem Kontakt mit einem Wärmebad der Temperatur $T$.

    \subsection{a)}
    Wie viele verschiedene 3-Teilchenzustände gibt es für die Fälle, dass die Teilchen
    \begin{enumerate}[label=(\roman*)]
        \item unterscheidbar
        \item Bosonen mit Spin 0
        \item Fermionen mit Spin ½
    \end{enumerate}
    sind?

    \subsection{b)}
    Berechnen Sie die kanonische Zustandssumme des Systems und seine innere Energie $U$ als Funktion von $\beta = \frac{1}{k_B T}$ für die drei Fälle.

    \subsection{c)}
    Berechnen Sie die Grenzfälle der inneren Energie für hohe und tiefe Temperaturen für die drei Fälle und interpretieren Sie die Ergebnisse physikalisch.
\end{aufgabe}

\end{document}
