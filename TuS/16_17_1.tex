\input{header_common.tex}

\subject{TuS}
\title{Thermodynamik und Statistik}
\date{
    Wintersemester 2016/2017
}

\begin{document}

\maketitle
\thispagestyle{empty}
\newpage


\begin{aufgabe}{Aufgabe 1: Kurzfragen}
    \subsection{a)}
    Betrachten Sie ein System $B$ das Sie in zwei Teile $B_1$ und $B_2$ aufteilen.
    Wenn sich die Systeme isoliert voneinander im Gleichgewicht befinden,
    wie setzt sich die Gesamtentropie $S$ des Systems $B$ aus den Teilentropien $S_1$ und $S_2$ zusammen?
    Was können Sie über die Gesamtentropie $S$ des Systems $B$ sagen,
    wenn die Teilsysteme gekoppelt werden?
    Wird sie größer, gleich oder kleiner sein als die Gesamtentropie ohne Kopplung?

    \subsection{b)}
    Wie lautet die Zustandsgleichung des idealen Gases?
    Welche Korrekturen werden für ein reales Gas benötigt?
    Geben Sie auch den physikalischen Ursprung der Korrekturterme an.


    \subsection{c)}
    Von welchen thermodynamischen Variablen hängt die freie Energie $F$ ab?
    Durch welche Transformation erhält man die innere Energie $U$ und von welchen Variablen hängt $U$ ab?
    Wie gewinnt man die Entropie $S$ und den Druck aus $F$?
    Ist $F$ eine extensive oder eine intensive Größe und warum?

    \subsection{d)}
    Wie lautet die mittlere Besetzung $n$ eines bosonischen Zustands mit der Energie $\epsilon$ bei der Temperatur $T$?
    Wann benötigen Sie für die Besetzung ein chemisches Potential?
    Wenn Sie die Dispersion des Lichts zugrunde legen,
    wie lautet dann die Energiedichte einer thermischen Lichtwelle (Plancksches Strahlungsgesetz)?

    \textit{Hinweis:} Summieren Sie alle $k$-Moden mit konstanter Energie auf und berücksichtigen Sie,
    dass das Licht zwei Polarisationsrichtungen besitzt.

    \subsection{e)}
    Wie berechnet man die kanonische Zustandssumme $Z$ für ein diskretes System?
    Geben Sie den Wert von $Z$ für ein System von $N$ unabhängigen Spins $s = \frac{1}{2}$ in einem endlichen externen Magnetfeld $B$ an.
    Wie lautet die Entropie $S$ des Systems im Grenzfall $B \to 0$?
    Woran sehen Sie, dass die Entropie extensiv ist?

    \subsection{f)}
    Wie unterscheiden sich die beiden spezifischen Wärmen $C_p$ und $C_v$?
    Wie lautet die Verknüpfung von Entropie und spezifischer Wärme?
    Aus welchen verschiedenen thermodynamischen Potentialen gewinnt man $C_v$ oder $C_p$ und warum?
\end{aufgabe}

\begin{aufgabe}{Aufgabe 2: Klassisches Gas im Potential}
    $N$ nicht miteinander wechselwirkende unterscheidbare Teilchen der Masse $m$ befinden sich im Inneren eines \textit{endlichen} Würfels mit Seitenlänge $L$ mit dem Potential
    \[
        V(\vec r) = \alpha(x + y + z), \quad \alpha > 0 \;.
    \]
    Der Würfel befindet sich im ersten Quadranten des Koordinatensystems und eine seiner Ecken berührt den Koordinatenursprung.

    \subsection{a)}
    Berechnen Sie die klassische kanonische Zustandssumme

    \subsection{b)}
    Berechnen Sie aus der kanonischen Zustandssumme die freie Energie $F$.
    Geben Sie den Zusammenhang zwischen der freien Energie $F$ und dem Druck $p$ an und berechnen Sie den Druck $p$.

    \subsection{c)}
    Wie lautet der Druck im Grenzfall $\alpha \to 0$?
    Entwickeln Sie dazu die Exponentialfunktion bis zur ersten Ordnung.
    Welche Zustandsgleichung ergibt sich?
\end{aufgabe}


\begin{aufgabe}{Aufgabe 3: Kreisprozess}
    Betrachten Sie folgenden Kreisprozess, der sich aus den vier angegebenen Teilprozessen zusammensetzt.
    Das Arbeitsmaterial ist ein ideales Gas.
    \begin{itemize}
        \item $1 \to 2$: adiabatische Kompression
        \item $2 \to 3$: isochore Druckerhöhung
        \item $3 \to 4$: adiabatische Expansion
        \item $4 \to 1$: isochore Druckverringerung
    \end{itemize}

    \subsection{a)}
    Zeichnen und beschriften Sie das $pV$- und das $TS$-Diagramm für diesen Kreisprozess.

    \textit{Hinweis:} Für adiabatische Prozesse gilt die Adiabatengleichung
    \[
        pV^\gamma = \text{const.} \;,
    \]
    wobei $\gamma = \frac{C_p}{C_V} = \frac{5}{3}$ für ein ideales Gas ist.

    \subsection{b)}
    Berechnen Sie für jeden Teilprozess die verrichtete Arbeit und Wärmeänderung.
    Sie brauchen nicht explizit angeben, ob dem Gas Wärme/Arbeit zu- oder abgeführt wird.

    \subsection{c)}
    Drücken Sie die gesamte Arbeit $W_\text{ges}(V_1, V_2, p_2, p_3)$ als Funktion der beiden Volumina sowie der beiden Drücke $p_2$ und $p_3$ aus.

    \subsection{d)}
    Berechnen Sie den Wirkungsgrad $\eta(V_1, V_2)$ als Funktion der beiden Volumina.
\end{aufgabe}

\begin{aufgabe}{Aufgabe 4: Drei-Zustand-System}
    Wir betrachten ein Teilchen in einem Zwei-Niveau-System mit den Energieniveaus $\epsilon_0$ und $\epsilon_1$ ($\epsilon_0 < \epsilon_1$).
    Der angeregte Zustand $\epsilon_1$ sei zweifach entartet und der Grundzustand $\epsilon_0$ sei nicht entartet.
    Der Hamiltonoperator enthalte keine Übergänge zwischen den beiden Niveaus.
    Das System stehe in Kontakt mit einem Wärmebad der Temperatur $T$.

    \subsection{a)}
    Geben Sie den Hamiltonoperator sowie den Dichteoperator im thermischen Gleichgewicht explizit als Matrizen an.


    Setzen Sie nun für die folgenden Aufgabenteile $\epsilon_0 = 0$ und $\epsilon_1 = \epsilon$.

    \subsection{b)}
    Beschreibt der Dichteoperator
    \begin{enumerate}[label=(\roman*)]
        \item im Grenzfall $T \to 0$
        \item im Grenzfall $T \to \infty$
    \end{enumerate}
    einen reinen Zustand?
    Interpretieren Sie Ihr Ergebnis physikalisch.

    \subsection{c)}
    Bestimmen Sie die innere Energie $U$ und die spezifische Wärme $C_V$.
    Skizzieren Sie den Verlauf der spezifischen Wärme $C_V$ in Abhängigkeit von der Temperatur $T$.
\end{aufgabe}


\end{document}
