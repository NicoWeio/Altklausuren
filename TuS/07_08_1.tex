\documentclass[
  bibliography=totoc,     % Literatur im Inhaltsverzeichnis
  captions=tableheading,  % Tabellenüberschriften
  titlepage=firstiscover, % Titelseite ist Deckblatt
]{scrartcl}

% Paket float verbessern
\usepackage{scrhack}

% Warnung, falls nochmal kompiliert werden muss
\usepackage[aux]{rerunfilecheck}

% unverzichtbare Mathe-Befehle
\usepackage{amsmath}
% viele Mathe-Symbole
\usepackage{amssymb}
% Erweiterungen für amsmath
\usepackage{mathtools}

% Fonteinstellungen
\usepackage{fontspec}
% Latin Modern Fonts werden automatisch geladen
% Alternativ zum Beispiel:
%\setromanfont{Libertinus Serif}
%\setsansfont{Libertinus Sans}
%\setmonofont{Libertinus Mono}

% Wenn man andere Schriftarten gesetzt hat,
% sollte man das Seiten-Layout neu berechnen lassen
\recalctypearea{}

% deutsche Spracheinstellungen
\usepackage[ngerman]{babel}


\usepackage[
  math-style=ISO,    % ┐
  bold-style=ISO,    % │
  sans-style=italic, % │ ISO-Standard folgen
  nabla=upright,     % │
  partial=upright,   % ┘
  warnings-off={           % ┐
    mathtools-colon,       % │ unnötige Warnungen ausschalten
    mathtools-overbracket, % │
  },                       % ┘
]{unicode-math}

% traditionelle Fonts für Mathematik
\setmathfont{Latin Modern Math}
% Alternativ zum Beispiel:
%\setmathfont{Libertinus Math}

% \setmathfont{XITS Math}[range={scr, bfscr}]
% \setmathfont{XITS Math}[range={cal, bfcal}, StylisticSet=1]

% Zahlen und Einheiten
\usepackage[
  locale=DE,                   % deutsche Einstellungen
  separate-uncertainty=true,   % immer Unsicherheit mit \pm
  per-mode=symbol-or-fraction, % / in inline math, fraction in display math
]{siunitx}

% chemische Formeln
\usepackage[
  version=4,
  math-greek=default, % ┐ mit unicode-math zusammenarbeiten
  text-greek=default, % ┘
]{mhchem}

% richtige Anführungszeichen
\usepackage[autostyle]{csquotes}

% schöne Brüche im Text
\usepackage{xfrac}

% Standardplatzierung für Floats einstellen
\usepackage{float}
\floatplacement{figure}{htbp}
\floatplacement{table}{htbp}

% Floats innerhalb einer Section halten
\usepackage[
  section, % Floats innerhalb der Section halten
  below,   % unterhalb der Section aber auf der selben Seite ist ok
]{placeins}

% Seite drehen für breite Tabellen: landscape Umgebung
\usepackage{pdflscape}

% Captions schöner machen.
\usepackage[
  labelfont=bf,        % Tabelle x: Abbildung y: ist jetzt fett
  font=small,          % Schrift etwas kleiner als Dokument
  width=0.9\textwidth, % maximale Breite einer Caption schmaler
]{caption}
% subfigure, subtable, subref
\usepackage{subcaption}

% Grafiken können eingebunden werden
\usepackage{graphicx}

% schöne Tabellen
\usepackage{booktabs}

% Verbesserungen am Schriftbild
\usepackage{microtype}

% Literaturverzeichnis
\usepackage[
  backend=biber,
]{biblatex}
% Quellendatenbank
\addbibresource{lit.bib}
% \addbibresource{programme.bib}

% Hyperlinks im Dokument
\usepackage[
  german,
  unicode,        % Unicode in PDF-Attributen erlauben
  pdfusetitle,    % Titel, Autoren und Datum als PDF-Attribute
  pdfcreator={},  % ┐ PDF-Attribute säubern
  pdfproducer={}, % ┘
]{hyperref}
% erweiterte Bookmarks im PDF
\usepackage{bookmark}

% Trennung von Wörtern mit Strichen
\usepackage[shortcuts]{extdash}

\usepackage{parskip}

\author{%
  Nicolai Weitkemper\\%
  \href{mailto:nicolai.weitkemper@tu-dortmund.de}{nicolai.weitkemper@tu-dortmund.de}%
}
\publishers{TU Dortmund – Fakultät Physik}

\global\def\colvec#1{\begin{pmatrix}#1\end{pmatrix}}
% \global\def\b#1{\left(#1\right)}
% \global\def\pdv#1{\frac{\partial}{\partial #1}}
\global\def\dv#1#2{\frac{\d #1}{\d #2}}
\global\def\pdv#1#2{\frac{\partial #1}{\partial #2}}
\global\def\pdvfest#1#2#3{\left.\frac{\partial #1}{\partial #2}\right\rvert_{#3}}
\global\def\d{\mathrm{d}}
\global\def\intfty{\int_{-\infty}^\infty} % ha ha
\global\def\half{\frac{1}{2}}
\global\def\quarter{\frac{1}{4}}

\global\def\b#1{\left(#1\right)}
\global\def\L{\mathcal{L}}
\global\def\normtwo{\frac{1}{\sqrt{2}}}


\global\def\kB{k_\text{B}}
\global\def\b#1{\left(#1\right)}
\global\def\a#1{\left\langle#1\right\rangle}

% https://tex.stackexchange.com/a/444226
\makeatletter
\renewcommand{\@seccntformat}[1]{}
\makeatother

\NewCommandCopy{\oldDelta}{\Delta}
\renewcommand{\Delta}{\mathrm{\oldDelta}}

\newenvironment{aufgabe}[1]
    {
    \section{#1}
    }
    {
    \clearpage
    }

% TODO: Nicht die neueste Version des Standard-header_common!


\subject{TuS}
\title{Thermodynamik und Statistik}
\date{
    Wintersemester 2007/2008
}

\begin{document}

\maketitle
\thispagestyle{empty}
% \tableofcontents
\newpage

\begin{aufgabe}{Aufgabe 1: Kurzfragen}
    \subsection{a)}
    Was besagt das Wiederkehrtheorem?

    Inwiefern scheint es der makrosopischen Erfahrung,
    dass Systeme sich einem thermischen Gleichgewicht annähern,
    zu widersprechen?

    Wie löst sich der scheinbare Widerspruch auf?


    \subsection{b)}
    Inwieweit scheint die mikroskopische Reversibilität der Bewegungsgleichungen der makrosopischen Erfahrung,
    dass Systeme sich einem thermischen Gleichgewicht annähern,
    zu widersprechen?

    Geben Sie zwei Überlegungen an,
    wie sich dieser scheinbare Widerspruch auflöst.

    \subsection{c)}
    Worin besteht in der klassischen Statistik das Gibbs'sche Paradoxon?

    Wie löst es sich in der Quantenmechanik auf?

    \subsection{d)}
    Was ist die Boltzmann-Entropie?

    Was ist die Informationsentropie?

    Wann werden die beiden Größen gleich?

    \subsection{e)}
    Erklären Sie die Begriffe extensive und intensive Zustandsgröße.
    Was gilt für das Vorzeichen der zweiten Ableitungen thermodynamischer Potentiala nach extensiven oder nach intensiven Zustandsgrößen und mit welcher Begründung?

    \subsection{f)}
    Benennen Sie die drei Üblichen Gesamtheiten (Ensembles) und geben Sie jeweils die zugehörige thermodynamische Größe an,
    die bei diesen Gesamtheiten minimal oder maximal wird.

    Welche natürlichen Variablen haben diese Größen jeweils?
\end{aufgabe}

\begin{aufgabe}{Aufgabe 2: Auf noch einer anderen Welt}
    Betrachten Sie in drei Dimensionen ein System von $N$ wechselwirkungsfreien, ununterscheidbaren Teilchen, deren Energie durch
    \[
        \epsilon(\vec r, \vec p) = \frac{\vec p^2}{2m} + V(z)
    \]
    gegeben ist.
    Das Potential $V(z)$ sei
    \[
        V(z) = k\sqrt{z}
    \]
    für $0 \leq z \leq \infty$ und mit $k > 0$.

    \subsection{a)}
    Leiten Sie die Einteilchenzustandssumme $Z_1$ als Funktion der thermischen de-Broglie-Wellenlänge $\lambda_\beta$ her.
    Wie lautet die Zustandssumme des gesamten Systems?

    \subsection{b)}
    Bestimmen Sie daraus die mittlere Gesamtenergie.

    \subsection{c)}
    Bestimmen Sie die Entropie $S$ und die freie Energie $F$.

    \textit{Hinweis:} Zur Berechnung des Zustandssummenintegrals substituieren Sie $\beta k \sqrt{z} = u$.
\end{aufgabe}

\begin{aufgabe}{Aufgabe 3: Quantenmechanischer harmonischer Oszillator in 2D}
    Der Hamilton-Operator eines harmonischen Oszillators in zwei Dimensionen sei in zweiter Quantisierung durch
    \[
        H = \hbar\omega_0 \sum_{i=0}^{2}{\b{b_i^\dagger b_i + \frac{1}{2}}}
    \]
    gegeben.

    \subsection{a)}
    Begründen Sie, dass zur Energie $E_m = (m+1)\hbar\omega_0$ der Entartungsgrad durch
    \[
        g(m) = m + 1
    \]
    gegeben ist.

    \subsection{b)}
    Berechnen Sie nun im Limes $\hbar\omega_0 \ll E$ die Zustandssumme
    \[
        Z = \int_0^\infty{D(E) \exp(-\beta E) \d E} \;.
    \]
    Dafür benötigen Sie die Zustandsdichte
    \[
        D(E := \sum_{m=0}^\infty{\delta\b{E - (m+1) \hbar\omega_0} g(m)} \;.
    \]
    Sie können $D(E)$ am einfachsten berechnen,
    wenn Sie die Summe in ein Integral umschreiben,
    wofür Sie $\hbar\omega_0 \ll E$ ausnutzen sollen.

    \subsection{c)}
    Berechnen Sie die Entropie $S$ f+ür $\hbar\omega_0 \ll k_B T$.
\end{aufgabe}

\begin{aufgabe}{Aufgabe 4: Informationsentropie}
    Die allgemeine Rechnung für ein quantenmechanisches System kann leicht auf ein Zwei-Zustands-System mit Energien $\epsilon_1$ und $\epsilon_2$ spezialisiert werden,
    bzw. auf ein $S = \frac{\hbar}{2}$-System.

    \subsection{a)}
    Bestimmen Sie – analog zur Vorlesung – die Wahrscheinlichkeiten $p_i$ so,
    dass die Entropie unter den Nebenbedingungen
    \begin{itemize}
        \item $\displaystyle \sum_{i=1}^2{p_i} = 1$
        \item und konstanter mittlerer Energie $\tilde\epsilon$
    \end{itemize}
    maximiert wird.
    Was bedeutet die Maximierung der Entropie?

    \subsection{b)}
    Der Hamiltonoperator des Zweiniveausystems sei durch $\hat H = \mu B \hat S_z$ gegeben.
    Dabei ist $\mu$ das magnetische Moment,
    $B$ das Magnetfeld
    und $S_z$ die $z$-Komponente des Spinoperators.
    Identifizieren Sie die Energien $\epsilon_i$ und bestimmen Sie die freie Energie $F$.

    \subsection{c)}
    Geben Sie die Magnetisierung $M = \pdvfest{F}{B}{T}$ als Funktion von $\beta$, $\mu$ und $B$ an.

    \subsection{d)}
    Berechnen Sie die Suszeptibilität $\chi = \pdvfest{M}{B}{B=0}$.
\end{aufgabe}

\begin{aufgabe}{Aufgabe 5: Dichteoperatoren}
    Für ein Drei-Zustandssystem mit Hamiltonoperator
    \[
        H = \colvec{-\epsilon & 0 & 0 \\ 0 & 0 & 0 \\ 0 & 0 & \epsilon}
    \]
    seien zur Zeit $t = 0$ die folgenden Dichteoperatoren gegeben:
    \begin{align*}
        p_\alpha &= \frac{1}{3}\colvec{1 & 0 & 0 \\ 0 & 1 & 0 \\ 0 & 0 & 1} &
        p_\beta  &= \frac{1}{2}\colvec{1 & 1 & 0 \\ 1 & 1 & 0 \\ 0 & 0 & 0} &
        p_\gamma &= \frac{1}{3}\colvec{1 & 1 & 0 \\ 1 & 1 & 0 \\ 0 & 0 & 1} \;.
    \end{align*}

    \subsection{a)}
    Geben Sie für jeden der drei Dichteoperatoren an,
    ob er
    i) \textit{rein}
    und
    ii) \textit{zeitlich konstant}
    ist.
    Begründen Sie Ihre Angaben.

    \subsection{b)}
    Berechnen Sie für alle drei Fälle $\a{H}$ und $\a{H^2}$ für $t=0$.

    \subsection{c)}
    Hängen die Erwartungswerte $\a{H}$ und $\a{H^2}$ im Allgemeinen von $t$ ab?
    (Begründung!)
\end{aufgabe}

\begin{aufgabe}{Aufgabe 6: Thermodynamische Relationen}
    Leiten Sie die folgenden thermodynamischen Relationen ab:
    \subsection{a)}
    \[
        \pdvfest{S}{V}{T} = \pdvfest{p}{T}{V}
    \]

    \subsection{b)}
    \[
        \pdvfest{E}{T}{V} = T \pdvfest{S}{T}{V}
    \]

    \subsection{c)}
    \[
        \frac{\kappa_S}{\kappa_T} = \frac{C_V}{C_p}
    \]

    \subsection{d)}
    \[
        C_p - C_v = \frac{T V \alpha^2}{\kappa_T}
    \]
\end{aufgabe}

\end{document}
