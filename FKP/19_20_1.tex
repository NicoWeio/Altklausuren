\documentclass[
  bibliography=totoc,     % Literatur im Inhaltsverzeichnis
  captions=tableheading,  % Tabellenüberschriften
  titlepage=firstiscover, % Titelseite ist Deckblatt
]{scrartcl}

% Paket float verbessern
\usepackage{scrhack}

% Warnung, falls nochmal kompiliert werden muss
\usepackage[aux]{rerunfilecheck}

% unverzichtbare Mathe-Befehle
\usepackage{amsmath}
% viele Mathe-Symbole
\usepackage{amssymb}
% Erweiterungen für amsmath
\usepackage{mathtools}

% Fonteinstellungen
\usepackage{fontspec}
% Latin Modern Fonts werden automatisch geladen
% Alternativ zum Beispiel:
%\setromanfont{Libertinus Serif}
%\setsansfont{Libertinus Sans}
%\setmonofont{Libertinus Mono}

% Wenn man andere Schriftarten gesetzt hat,
% sollte man das Seiten-Layout neu berechnen lassen
\recalctypearea{}

% deutsche Spracheinstellungen
\usepackage[ngerman]{babel}


\usepackage[
  math-style=ISO,    % ┐
  bold-style=ISO,    % │
  sans-style=italic, % │ ISO-Standard folgen
  nabla=upright,     % │
  partial=upright,   % ┘
  warnings-off={           % ┐
    mathtools-colon,       % │ unnötige Warnungen ausschalten
    mathtools-overbracket, % │
  },                       % ┘
]{unicode-math}

% traditionelle Fonts für Mathematik
\setmathfont{Latin Modern Math}
% Alternativ zum Beispiel:
%\setmathfont{Libertinus Math}

% \setmathfont{XITS Math}[range={scr, bfscr}]
% \setmathfont{XITS Math}[range={cal, bfcal}, StylisticSet=1]

% Zahlen und Einheiten
\usepackage[
  locale=DE,                   % deutsche Einstellungen
  separate-uncertainty=true,   % immer Unsicherheit mit \pm
  per-mode=symbol-or-fraction, % / in inline math, fraction in display math
]{siunitx}

% chemische Formeln
\usepackage[
  version=4,
  math-greek=default, % ┐ mit unicode-math zusammenarbeiten
  text-greek=default, % ┘
]{mhchem}

% richtige Anführungszeichen
\usepackage[autostyle]{csquotes}

% schöne Brüche im Text
\usepackage{xfrac}

% Standardplatzierung für Floats einstellen
\usepackage{float}
\floatplacement{figure}{htbp}
\floatplacement{table}{htbp}

% Floats innerhalb einer Section halten
\usepackage[
  section, % Floats innerhalb der Section halten
  below,   % unterhalb der Section aber auf der selben Seite ist ok
]{placeins}

% Seite drehen für breite Tabellen: landscape Umgebung
\usepackage{pdflscape}

% Captions schöner machen.
\usepackage[
  labelfont=bf,        % Tabelle x: Abbildung y: ist jetzt fett
  font=small,          % Schrift etwas kleiner als Dokument
  width=0.9\textwidth, % maximale Breite einer Caption schmaler
]{caption}
% subfigure, subtable, subref
\usepackage{subcaption}

% Grafiken können eingebunden werden
\usepackage{graphicx}

% schöne Tabellen
\usepackage{booktabs}

% Verbesserungen am Schriftbild
\usepackage{microtype}

% Literaturverzeichnis
\usepackage[
  backend=biber,
]{biblatex}
% Quellendatenbank
\addbibresource{lit.bib}
% \addbibresource{programme.bib}

% Hyperlinks im Dokument
\usepackage[
  german,
  unicode,        % Unicode in PDF-Attributen erlauben
  pdfusetitle,    % Titel, Autoren und Datum als PDF-Attribute
  pdfcreator={},  % ┐ PDF-Attribute säubern
  pdfproducer={}, % ┘
]{hyperref}
% erweiterte Bookmarks im PDF
\usepackage{bookmark}

% Trennung von Wörtern mit Strichen
\usepackage[shortcuts]{extdash}

\usepackage{parskip}

\author{%
  Nicolai Weitkemper\\%
  \href{mailto:nicolai.weitkemper@tu-dortmund.de}{nicolai.weitkemper@tu-dortmund.de}%
}
\publishers{TU Dortmund – Fakultät Physik}

\global\def\colvec#1{\begin{pmatrix}#1\end{pmatrix}}
% \global\def\b#1{\left(#1\right)}
% \global\def\pdv#1{\frac{\partial}{\partial #1}}
\global\def\dv#1#2{\frac{\d #1}{\d #2}}
\global\def\pdv#1#2{\frac{\partial #1}{\partial #2}}
\global\def\pdvfest#1#2#3{\left.\frac{\partial #1}{\partial #2}\right\rvert_{#3}}
\global\def\d{\mathrm{d}}
\global\def\intfty{\int_{-\infty}^\infty} % ha ha
\global\def\half{\frac{1}{2}}
\global\def\quarter{\frac{1}{4}}

\global\def\b#1{\left(#1\right)}
\global\def\L{\mathcal{L}}
\global\def\normtwo{\frac{1}{\sqrt{2}}}


\global\def\kB{k_\text{B}}
\global\def\b#1{\left(#1\right)}
\global\def\a#1{\left\langle#1\right\rangle}

% https://tex.stackexchange.com/a/444226
\makeatletter
\renewcommand{\@seccntformat}[1]{}
\makeatother

\NewCommandCopy{\oldDelta}{\Delta}
\renewcommand{\Delta}{\mathrm{\oldDelta}}

\newenvironment{aufgabe}[1]
    {
    \section{#1}
    }
    {
    \clearpage
    }

% TODO: Nicht die neueste Version des Standard-header_common!


\subject{FKP}
\title{Festkörperphysik}
\date{
    Wintersemester 2019/2020
}

\begin{document}

\maketitle
\thispagestyle{empty}
\newpage


\begin{aufgabe}{Aufgabe 1: Kurzfragen}
    \subsection{a)}
    Wie ist die 1. Brillouin-Zone definiert?
    Geben Sie zwei Beispiele für die Bedeutung des Brillouin-Zonenrands oder Phänomene, die dort auftreten, an.

    \subsection{b)}
    Beschreiben Sie kurz, was man unter dem Meißner-Ochsenfeld-Effekt versteht.
    Tragen Sie für Typ-I und II-Supraleiter außerdem die Suszeptibilität gegen das äußere $B$-Feld auf.

    % \subsection{c)}
    % In einem 2-dimensionalen Gitter,
    % die Quadrate stellen die 1. Brillouin-Zone dar,
    % finden die angedeuteten Dreiphononenprozesse statt.
    % Vervollständigen Sie die Skizze, geben Sie den Energie- und Impulserhaltungssatz an.
    % Benennen Sie die jeweiligen Prozesse.
    %
    % TODO: Skizze

    \subsection{d)}
    Was bezeichnet eine Van-Hove-Singularität?
    Wie hängt diese mit einer Dispersionsrelation zusammen und wie kann man sie messen?

    % \subsection{e)}
    % Zeichnen Sie in den untenstehenden Graphen die inverse Suszeptibilität in Abhängigkeit der Temperatur für Paramagnetismus, Ferromagnetismus und Antiferromagnetismus.
    % Kennzeichnen Sie dabei die auftretenden Temperaturen und benennen Sie diese.
    %
    % TODO: Skizze

    \subsection{f)}
    Erklären Sie kurz, was man unter einem Exziton, einem Magnon und einem Plasmon versteht.
\end{aufgabe}

\begin{aufgabe}{Aufgabe 2: Aus Laue mach Bragg}
    \subsection{a)}
    Zeigen Sie, dass aus der Lauebedingung im Fall elastischer Streuung die Braggbedingung folgt.
    Fertigen Sie hierzu eine Skizze an und tragen Sie die nötigen Größen ein.

    \subsection{b)}
    Geben Sie in Abhängigkeit von
    der Wellenlänge $\lambda$ der Röntgenstrahlung,
    der Gitterkonstanten $a$
    und der Millerindices $h, k, l$
    die fünf niedrigsten Werte für $\sin\theta$ an,
    bei denen Röntgenreflexe erster Ordnung am einfachen kubischen Gitter auftreten.
    Der Braggwinkel wird als $\theta$ bezeichnet.

    \subsection{c)}
    Berechnen Sie den Strukturfaktor eines bcc-Gitters und zeigen Sie anhand der Auswahlregeln,
    welche der in \textbf{b)} angegebenen Interferenzlinien in einem bcc-Gitter erlaubt sind.
\end{aufgabe}

% TODO: Aufgabe 3

\begin{aufgabe}{Aufgabe 4: Fermigase im Weltall}
    Betrachten Sie einen Atomkern als ein System unabhängiger Fermionen mit Spin $s = \frac{1}{2}$.

    \subsection{a)}
    Das Volumen eines Atomkerns mit der Nukleonenzahl $N$ lässt sich in guter Näherung mittels der Formel
    $V = N \cdot \frac{4}{3} \cdot \pi \cdot r_0^3$
    berechnen,
    wobei $r_0 \approx \SI{1.2}{\femto\meter}$ den Radius eines Nukleons darstellt.
    Zeigen Sie, dass nach diesem Modell eine Fermi-Energie von etwa $\SI{53}{\mega\electronvolt}$
    bei gegebenem Betrach des Fermi-Vektors $k_F = (3\pi^2 n)^{1/3}$ mit der Teilchendichte $n$
    zu erwarten ist.
    Die Protonen und Neutronen im Kern sollen dabei die gleiche Masse $m = \SI{1.675E-27}{\kilo\gram}$ besitzen.


    Betrachten Sie als Nächstes einen kugelförmigen Zwergstern mit dem Radius $R = \SI{5}{\kilo\meter}$,
    der als großer Atomkern bestehend aus Neutronen der Massen $m$ modelliert werden kann.

    \subsection{b)}
    Berechnen Sie den Druck $p = - \pdvfest{U}{V}{N}$ und den Kompressibilitätsmodul $\kappa = -V \pdvfest{p}{V}{T}$ des Fermigases aus Neutronen für $T = \SI{0}{\kelvin}$.
    Nutzen Sie dafür die Zustandsdichte in drei Dimensionen $D(E) = \frac{V}{4\pi^2} \b{\frac{2m}{\hbar^2}}^{3/2} E^{1/2}$.

    \subsection{c)}
    Berechnen Sie
    unter Verwendung der Fermi-Energie aus Aufgabenteil \textbf{a)} und der Zustandsdichte in drei Dimensionen
    die Gesamtenergie $U$ des Neutronensterns für $T = \SI{0}{\kelvin}$.

    \subsection{d)}
    Für Neutronen hoher Energie und bei endlichen Temperaturen lässt sich die Fermi-Dirac-Statistik vereinfachen zu
    \[
        f(E) \approx \frac{N}{V} \b{2\pi\hbar^2}{m k_B T}^{3/2} \cdot e^{-E / k_B T} \;.
    \]
    Zeigen Sie, dass in diesem Grenzfall die innere Energie $U$ des Neutronensterns der eines idealen Gases entspricht.
\end{aufgabe}

\end{document}
