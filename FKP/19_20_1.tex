\input{header_common.tex}

\subject{FKP}
\title{Festkörperphysik}
\date{
    Wintersemester 2019/2020
}

\begin{document}

\maketitle
\thispagestyle{empty}
\newpage


\begin{aufgabe}{Aufgabe 1: Kurzfragen}
    \subsection{a)}
    Wie ist die 1. Brillouin-Zone definiert?
    Geben Sie zwei Beispiele für die Bedeutung des Brillouin-Zonenrands oder Phänomene, die dort auftreten, an.

    \subsection{b)}
    Beschreiben Sie kurz, was man unter dem Meißner-Ochsenfeld-Effekt versteht.
    Tragen Sie für Typ-I und II-Supraleiter außerdem die Suszeptibilität gegen das äußere $B$-Feld auf.

    % \subsection{c)}
    % In einem 2-dimensionalen Gitter,
    % die Quadrate stellen die 1. Brillouin-Zone dar,
    % finden die angedeuteten Dreiphononenprozesse statt.
    % Vervollständigen Sie die Skizze, geben Sie den Energie- und Impulserhaltungssatz an.
    % Benennen Sie die jeweiligen Prozesse.
    %
    % TODO: Skizze

    \subsection{d)}
    Was bezeichnet eine Van-Hove-Singularität?
    Wie hängt diese mit einer Dispersionsrelation zusammen und wie kann man sie messen?

    % \subsection{e)}
    % Zeichnen Sie in den untenstehenden Graphen die inverse Suszeptibilität in Abhängigkeit der Temperatur für Paramagnetismus, Ferromagnetismus und Antiferromagnetismus.
    % Kennzeichnen Sie dabei die auftretenden Temperaturen und benennen Sie diese.
    %
    % TODO: Skizze

    \subsection{f)}
    Erklären Sie kurz, was man unter einem Exziton, einem Magnon und einem Plasmon versteht.
\end{aufgabe}

\begin{aufgabe}{Aufgabe 2: Aus Laue mach Bragg}
    \subsection{a)}
    Zeigen Sie, dass aus der Lauebedingung im Fall elastischer Streuung die Braggbedingung folgt.
    Fertigen Sie hierzu eine Skizze an und tragen Sie die nötigen Größen ein.

    \subsection{b)}
    Geben Sie in Abhängigkeit von
    der Wellenlänge $\lambda$ der Röntgenstrahlung,
    der Gitterkonstanten $a$
    und der Millerindices $h, k, l$
    die fünf niedrigsten Werte für $\sin\theta$ an,
    bei denen Röntgenreflexe erster Ordnung am einfachen kubischen Gitter auftreten.
    Der Braggwinkel wird als $\theta$ bezeichnet.

    \subsection{c)}
    Berechnen Sie den Strukturfaktor eines bcc-Gitters und zeigen Sie anhand der Auswahlregeln,
    welche der in \textbf{b)} angegebenen Interferenzlinien in einem bcc-Gitter erlaubt sind.
\end{aufgabe}

% TODO: Aufgabe 3

\begin{aufgabe}{Aufgabe 4: Fermigase im Weltall}
    Betrachten Sie einen Atomkern als ein System unabhängiger Fermionen mit Spin $s = \frac{1}{2}$.

    \subsection{a)}
    Das Volumen eines Atomkerns mit der Nukleonenzahl $N$ lässt sich in guter Näherung mittels der Formel
    $V = N \cdot \frac{4}{3} \cdot \pi \cdot r_0^3$
    berechnen,
    wobei $r_0 \approx \SI{1.2}{\femto\meter}$ den Radius eines Nukleons darstellt.
    Zeigen Sie, dass nach diesem Modell eine Fermi-Energie von etwa $\SI{53}{\mega\electronvolt}$
    bei gegebenem Betrach des Fermi-Vektors $k_F = (3\pi^2 n)^{1/3}$ mit der Teilchendichte $n$
    zu erwarten ist.
    Die Protonen und Neutronen im Kern sollen dabei die gleiche Masse $m = \SI{1.675E-27}{\kilo\gram}$ besitzen.


    Betrachten Sie als Nächstes einen kugelförmigen Zwergstern mit dem Radius $R = \SI{5}{\kilo\meter}$,
    der als großer Atomkern bestehend aus Neutronen der Massen $m$ modelliert werden kann.

    \subsection{b)}
    Berechnen Sie den Druck $p = - \pdvfest{U}{V}{N}$ und den Kompressibilitätsmodul $\kappa = -V \pdvfest{p}{V}{T}$ des Fermigases aus Neutronen für $T = \SI{0}{\kelvin}$.
    Nutzen Sie dafür die Zustandsdichte in drei Dimensionen $D(E) = \frac{V}{4\pi^2} \b{\frac{2m}{\hbar^2}}^{3/2} E^{1/2}$.

    \subsection{c)}
    Berechnen Sie
    unter Verwendung der Fermi-Energie aus Aufgabenteil \textbf{a)} und der Zustandsdichte in drei Dimensionen
    die Gesamtenergie $U$ des Neutronensterns für $T = \SI{0}{\kelvin}$.

    \subsection{d)}
    Für Neutronen hoher Energie und bei endlichen Temperaturen lässt sich die Fermi-Dirac-Statistik vereinfachen zu
    \[
        f(E) \approx \frac{N}{V} \b{2\pi\hbar^2}{m k_B T}^{3/2} \cdot e^{-E / k_B T} \;.
    \]
    Zeigen Sie, dass in diesem Grenzfall die innere Energie $U$ des Neutronensterns der eines idealen Gases entspricht.
\end{aufgabe}

\end{document}
