\documentclass[
  bibliography=totoc,     % Literatur im Inhaltsverzeichnis
  captions=tableheading,  % Tabellenüberschriften
  titlepage=firstiscover, % Titelseite ist Deckblatt
]{scrartcl}

% Paket float verbessern
\usepackage{scrhack}

% Warnung, falls nochmal kompiliert werden muss
\usepackage[aux]{rerunfilecheck}

% unverzichtbare Mathe-Befehle
\usepackage{amsmath}
% viele Mathe-Symbole
\usepackage{amssymb}
% Erweiterungen für amsmath
\usepackage{mathtools}

% Fonteinstellungen
\usepackage{fontspec}
% Latin Modern Fonts werden automatisch geladen
% Alternativ zum Beispiel:
%\setromanfont{Libertinus Serif}
%\setsansfont{Libertinus Sans}
%\setmonofont{Libertinus Mono}

% Wenn man andere Schriftarten gesetzt hat,
% sollte man das Seiten-Layout neu berechnen lassen
\recalctypearea{}

% deutsche Spracheinstellungen
\usepackage[ngerman]{babel}


\usepackage[
  math-style=ISO,    % ┐
  bold-style=ISO,    % │
  sans-style=italic, % │ ISO-Standard folgen
  nabla=upright,     % │
  partial=upright,   % ┘
  warnings-off={           % ┐
    mathtools-colon,       % │ unnötige Warnungen ausschalten
    mathtools-overbracket, % │
  },                       % ┘
]{unicode-math}

% traditionelle Fonts für Mathematik
\setmathfont{Latin Modern Math}
% Alternativ zum Beispiel:
%\setmathfont{Libertinus Math}

% \setmathfont{XITS Math}[range={scr, bfscr}]
% \setmathfont{XITS Math}[range={cal, bfcal}, StylisticSet=1]

% Zahlen und Einheiten
\usepackage[
  locale=DE,                   % deutsche Einstellungen
  separate-uncertainty=true,   % immer Unsicherheit mit \pm
  per-mode=symbol-or-fraction, % / in inline math, fraction in display math
]{siunitx}

% chemische Formeln
\usepackage[
  version=4,
  math-greek=default, % ┐ mit unicode-math zusammenarbeiten
  text-greek=default, % ┘
]{mhchem}

% richtige Anführungszeichen
\usepackage[autostyle]{csquotes}

% schöne Brüche im Text
\usepackage{xfrac}

% Standardplatzierung für Floats einstellen
\usepackage{float}
\floatplacement{figure}{htbp}
\floatplacement{table}{htbp}

% Floats innerhalb einer Section halten
\usepackage[
  section, % Floats innerhalb der Section halten
  below,   % unterhalb der Section aber auf der selben Seite ist ok
]{placeins}

% Seite drehen für breite Tabellen: landscape Umgebung
\usepackage{pdflscape}

% Captions schöner machen.
\usepackage[
  labelfont=bf,        % Tabelle x: Abbildung y: ist jetzt fett
  font=small,          % Schrift etwas kleiner als Dokument
  width=0.9\textwidth, % maximale Breite einer Caption schmaler
]{caption}
% subfigure, subtable, subref
\usepackage{subcaption}

% Grafiken können eingebunden werden
\usepackage{graphicx}

% schöne Tabellen
\usepackage{booktabs}

% Verbesserungen am Schriftbild
\usepackage{microtype}

% Literaturverzeichnis
\usepackage[
  backend=biber,
]{biblatex}
% Quellendatenbank
\addbibresource{lit.bib}
% \addbibresource{programme.bib}

% Hyperlinks im Dokument
\usepackage[
  german,
  unicode,        % Unicode in PDF-Attributen erlauben
  pdfusetitle,    % Titel, Autoren und Datum als PDF-Attribute
  pdfcreator={},  % ┐ PDF-Attribute säubern
  pdfproducer={}, % ┘
]{hyperref}
% erweiterte Bookmarks im PDF
\usepackage{bookmark}

% Trennung von Wörtern mit Strichen
\usepackage[shortcuts]{extdash}

\usepackage{parskip}

\author{%
  Nicolai Weitkemper\\%
  \href{mailto:nicolai.weitkemper@tu-dortmund.de}{nicolai.weitkemper@tu-dortmund.de}%
}
\publishers{TU Dortmund – Fakultät Physik}

\global\def\colvec#1{\begin{pmatrix}#1\end{pmatrix}}
% \global\def\b#1{\left(#1\right)}
% \global\def\pdv#1{\frac{\partial}{\partial #1}}
\global\def\dv#1#2{\frac{\d #1}{\d #2}}
\global\def\pdv#1#2{\frac{\partial #1}{\partial #2}}
\global\def\pdvfest#1#2#3{\left.\frac{\partial #1}{\partial #2}\right\rvert_{#3}}
\global\def\d{\mathrm{d}}
\global\def\intfty{\int_{-\infty}^\infty} % ha ha
\global\def\half{\frac{1}{2}}
\global\def\quarter{\frac{1}{4}}

\global\def\b#1{\left(#1\right)}
\global\def\L{\mathcal{L}}
\global\def\normtwo{\frac{1}{\sqrt{2}}}


\global\def\kB{k_\text{B}}
\global\def\b#1{\left(#1\right)}
\global\def\a#1{\left\langle#1\right\rangle}

% https://tex.stackexchange.com/a/444226
\makeatletter
\renewcommand{\@seccntformat}[1]{}
\makeatother

\NewCommandCopy{\oldDelta}{\Delta}
\renewcommand{\Delta}{\mathrm{\oldDelta}}

\newenvironment{aufgabe}[1]
    {
    \section{#1}
    }
    {
    \clearpage
    }

% TODO: Nicht die neueste Version des Standard-header_common!


\subject{FKP}
\title{Festkörperphysik}
\date{
    Wintersemester 2015/2016
}

\begin{document}

\maketitle
\thispagestyle{empty}
\newpage


\begin{aufgabe}{Aufgabe 1: Kurzfragen}
    \subsection{a)}
    Konstruieren Sie die Wigner-Seitz-Zelle des untenstehenden Gitters.
    % TODO
    \subsection{b)}
    Skizzieren Sie den Debye-Waller-Faktor für Beugungsreflexe in Abhängigkeit der Temperatur für einen kleinen und einen großen Streuwinkel.
    Wie beeinflusst dieser das Röntgenbeugungsbild? % Scan hier abgeschnitten
    Welche physikalische Größe kann durch die Messung des Debye-Waller-Faktors bestimmt werden?
    \subsection{c)}
    Nennen Sie drei Verfahren zur Bestimmung der Fermifläche.
    \subsection{d)}
    Erläutern Sie den Unterschied zwischen einem direkten und einem indirekten Halbleiter.
    \subsection{e)}
    Formulieren Sie das Blochtheorem.
    Was besagt dieses?
    \subsection{f)}
    Nennen Sie drei mit Forschernamen belegte Arten des Festkörpermagnetismus und beschreiben Sie kurz die jeweils zugrunde liegenden mikroskopischen Ursachen.
    \subsection{g)}
    Was versteht man unter dem Meißner-Ochsenfeld-Effekt?
    \subsection{h)}
    Welche Typen von Domänenwänden gibt es und worin unterscheiden sich diese?
    Welche Energiebeiträge bestimmen die Dicke von Domänenwänden?
\end{aufgabe}

\begin{aufgabe}{Aufgabe 2: Debye-Scherrer}
    Gegeben sei eine Debye-Scherrer-Aufnahme von Palladium.
    Dieses besitzt eine Gitterkonstante von $a = \SI{3.77}{\angstrom}$
    und die Röntgenstrahlungsquelle emittiert bei einer mittleren Wellenlänge von $\lambda = \SI{1.54}{\angstrom}$.
    Es sind folgende Reflexe beobachtet worden:
    \[
        \Theta =
            \SI{20.01}{\degree};
            \SI{23.4}{\degree};
            \SI{34.0}{\degree};
            \SI{41.2}{\degree};
            \SI{43.5}{\degree}
    \]

    \subsection{a)}
    Machen Sie den Versuchsaufbau eines Debye-Scherrer-Versuches anhand einer beschrifteten Skizze klar.

    \subsection{b)}
    Zeigen Sie durch explizite Rechnung des Strukturfaktors,
    dass für das fcc-Gitter folgende Auslöschregeln gelten:
    \[
        S = \begin{cases}
            4f & \text{$h,k,l$ gerade oder ungerade} \\
            0 & \text{sonst}
        \end{cases}
    \]

    \subsection{c)}
    Verwenden Sie diese Auslöschregeln,
    um anhand von mindestens drei der oben genannten Reflexe die fcc-Gitterstruktur von Palladium nachzuweisen.
    Nutzen Sie dafür die Braggsche Beugungsbedingung für die Reflexe erster Ordnung und den Netzebenenabstand $d_{h,k,l}$.

    \subsection{d)}
    Beschreiben Sie, wie sich die Auslöschregeln ändern,
    wenn ein Diamant-Gitter anstatt eines fcc-Gitters vorliegen würde.
\end{aufgabe}

\begin{aufgabe}{Aufgabe 4: Ionische Bindung}
    Die potentielle Energie von Ionen im Kristall setzt sich zusammen aus der bindenden Coulombenergie und einem abstoßenden Anteil,
    für den hier das exponentiell verlaufende Potential nach dem Ansatz von Born und Mayer angenommen werden soll:
    \[
        U = N \b{-\frac{\alpha e^2}{4\pi\epsilon_0r} + B\exp\bb{-\frac{r}{\rho}}}
    \]
    Im Gleichgewicht beträgt der mittlere Abstand zum nächsten Nachbarn $r_0$ und der resultierende Druck $P = - \pdvfest{U}{V}{r=r_0} = 0$ verschwindet.
    Für das Volumen soll $V = Nr^3$ gelten.

    \subsection{a)}
    Welche physikalische Bedeutung haben $\alpha$ und $\rho$?

    \subsection{b)}
    Zeigen Sie, dass für die Konstante $B$ im Gleichgewichtsabstand $r_0$ gilt:
    \[
        B = \frac{\alpha e^2}{4\pi\epsilon_0} \frac{\rho\exp\b{\frac{r_0}{\rho}}}{r_0^2}
    \]

    \subsection{c)}
    Berechnen Sie den Kompressionsmodul $K = -V \pdvfest{P}{V}{r=r_0}$.

    Tipp: Benutzen Sie $\pdv{P}{V} = \pdv{r}{V}\pdv{P}{r}$.
\end{aufgabe}

\begin{aufgabe}{Aufgabe 6: Landau-Theorie}
    Die Freie Energie zu einer Magnetisierung $m$ werde ohne ein externes Magnetfeld nach der Landau-Theorie mit
    \[
        F(m) = F_0 + rm^2 + sm^4
    \]
    genähert, wobei $s > 0$ sein muss.

    \subsection{a)}
    Warum gibt es keine Terme, die proportional zu $m$ oder $m^3$ sind?
    Warum muss $s > 0$ gelten?
    Was bedeutet $F_0$?

    \subsection{b)}
    Bestimmen Sie die Minima von $F(m)$.
    Was passiert, wenn $r$ das Vorzeichen wechselt?
    Was bedeuten die Minima?

    \subsection{c)}
    Im Folgenden werde $r = a \cdot (T - T_C)$ mit $a > 0$ genähert.
    Welcher kritische Exponent $\alpha$ ergibt sich für $m \propto (T - T_C)^\alpha$ daraus für $T < T_C$?

    \subsection{d)}
    Um einen Phasenübergang welcher Ordnung handelt es sich?
    Begründen Sie Ihre Antwort.
\end{aufgabe}


\end{document}
