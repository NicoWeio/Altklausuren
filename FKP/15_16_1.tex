\input{header_common.tex}

\subject{FKP}
\title{Festkörperphysik}
\date{
    Wintersemester 2015/2016
}

\begin{document}

\maketitle
\thispagestyle{empty}
\newpage


\begin{aufgabe}{Aufgabe 1: Kurzfragen}
    \subsection{a)}
    Konstruieren Sie die Wigner-Seitz-Zelle des untenstehenden Gitters.
    % TODO
    \subsection{b)}
    Skizzieren Sie den Debye-Waller-Faktor für Beugungsreflexe in Abhängigkeit der Temperatur für einen kleinen und einen großen Streuwinkel.
    Wie beeinflusst dieser das Röntgenbeugungsbild? % Scan hier abgeschnitten
    Welche physikalische Größe kann durch die Messung des Debye-Waller-Faktors bestimmt werden?
    \subsection{c)}
    Nennen Sie drei Verfahren zur Bestimmung der Fermifläche.
    \subsection{d)}
    Erläutern Sie den Unterschied zwischen einem direkten und einem indirekten Halbleiter.
    \subsection{e)}
    Formulieren Sie das Blochtheorem.
    Was besagt dieses?
    \subsection{f)}
    Nennen Sie drei mit Forschernamen belegte Arten des Festkörpermagnetismus und beschreiben Sie kurz die jeweils zugrunde liegenden mikroskopischen Ursachen.
    \subsection{g)}
    Was versteht man unter dem Meißner-Ochsenfeld-Effekt?
    \subsection{h)}
    Welche Typen von Domänenwänden gibt es und worin unterscheiden sich diese?
    Welche Energiebeiträge bestimmen die Dicke von Domänenwänden?
\end{aufgabe}

\begin{aufgabe}{Aufgabe 2: Debye-Scherrer}
    Gegeben sei eine Debye-Scherrer-Aufnahme von Palladium.
    Dieses besitzt eine Gitterkonstante von $a = \SI{3.77}{\angstrom}$
    und die Röntgenstrahlungsquelle emittiert bei einer mittleren Wellenlänge von $\lambda = \SI{1.54}{\angstrom}$.
    Es sind folgende Reflexe beobachtet worden:
    \[
        \Theta =
            \SI{20.01}{\degree};
            \SI{23.4}{\degree};
            \SI{34.0}{\degree};
            \SI{41.2}{\degree};
            \SI{43.5}{\degree}
    \]

    \subsection{a)}
    Machen Sie den Versuchsaufbau eines Debye-Scherrer-Versuches anhand einer beschrifteten Skizze klar.

    \subsection{b)}
    Zeigen Sie durch explizite Rechnung des Strukturfaktors,
    dass für das fcc-Gitter folgende Auslöschregeln gelten:
    \[
        S = \begin{cases}
            4f & \text{$h,k,l$ gerade oder ungerade} \\
            0 & \text{sonst}
        \end{cases}
    \]

    \subsection{c)}
    Verwenden Sie diese Auslöschregeln,
    um anhand von mindestens drei der oben genannten Reflexe die fcc-Gitterstruktur von Palladium nachzuweisen.
    Nutzen Sie dafür die Braggsche Beugungsbedingung für die Reflexe erster Ordnung und den Netzebenenabstand $d_{h,k,l}$.

    \subsection{d)}
    Beschreiben Sie, wie sich die Auslöschregeln ändern,
    wenn ein Diamant-Gitter anstatt eines fcc-Gitters vorliegen würde.
\end{aufgabe}

\begin{aufgabe}{Aufgabe 4: Ionische Bindung}
    Die potentielle Energie von Ionen im Kristall setzt sich zusammen aus der bindenden Coulombenergie und einem abstoßenden Anteil,
    für den hier das exponentiell verlaufende Potential nach dem Ansatz von Born und Mayer angenommen werden soll:
    \[
        U = N \b{-\frac{\alpha e^2}{4\pi\epsilon_0r} + B\exp\bb{-\frac{r}{\rho}}}
    \]
    Im Gleichgewicht beträgt der mittlere Abstand zum nächsten Nachbarn $r_0$ und der resultierende Druck $P = - \pdvfest{U}{V}{r=r_0} = 0$ verschwindet.
    Für das Volumen soll $V = Nr^3$ gelten.

    \subsection{a)}
    Welche physikalische Bedeutung haben $\alpha$ und $\rho$?

    \subsection{b)}
    Zeigen Sie, dass für die Konstante $B$ im Gleichgewichtsabstand $r_0$ gilt:
    \[
        B = \frac{\alpha e^2}{4\pi\epsilon_0} \frac{\rho\exp\b{\frac{r_0}{\rho}}}{r_0^2}
    \]

    \subsection{c)}
    Berechnen Sie den Kompressionsmodul $K = -V \pdvfest{P}{V}{r=r_0}$.

    Tipp: Benutzen Sie $\pdv{P}{V} = \pdv{r}{V}\pdv{P}{r}$.
\end{aufgabe}

\end{document}
