\input{header_common.tex}

\subject{FKP}
\title{Festkörperphysik}
\date{
    Wintersemester 2003/2004
}

\begin{document}

\maketitle
\thispagestyle{empty}
\newpage


\begin{aufgabe}{Aufgabe 1: Kurzfragen}
    \subsection{a)}
    Was beschreibt die Paarkorrelationsfunktion $g(r)$?
    Skizzieren Sie die Paarkorrelationsfunktion $g(r)$ eines amorphen und eines kristallinen Festkörpers.
    Welchen Einfluss hat eine endliche Temperatur auf die Breite eines Peaks in der Paarkorrelationsfunktion $g(r)$ eines Gitters?

    \subsection{b)}
    Wieviele Phononenzweige ergeben sich für $p$ Basisatome einer Elementarzelle in einem dreidimensionalen Gitter?

    \subsection{c)}
    Skizzieren Sie in dem abgebildeten Diagramm die spezifische Wärme eines Metalls (z.B. Kupfer) für niedrige Temperaturen.
    Welche Bedeutung hat der Achsenabschnitt ($T = 0$) im Diagramm?

    \subsection{d)}
    Geben Sie die Energieabhängigkeit der elektronischen Zustandsdichte in 0, 1, 2 und 3 Dimensionen an.

    \subsection{e)}
    Erläutern Sie knapp die Begriffe \enquote{extrinsische} und \enquote{intrinsische} Leitfähigkeit.

    \subsection{f)}
    Nennen Sie je 2 charakteristische Merkmale des Dia-, Para- und Ferromagnetismus.
\end{aufgabe}
\end{document}
