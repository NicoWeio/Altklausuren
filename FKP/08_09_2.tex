\input{header_common.tex}

\subject{FKP}
\title{Festkörperphysik}
\date{
    Wintersemester 2008/2009
}

\begin{document}

\maketitle
\thispagestyle{empty}
\newpage


\begin{aufgabe}{Aufgabe 1: Kurzfragen}
    \subsection{a)}
    Zeichnen und beschriften Sie eine typische Dispersionsrelation für Phononen.
    Nehmen Sie dazu einen Kristall mit zweiatomiger Einheitszelle an.
    Warum kann man sich auf die erste Brillouinzone beschränken?

    \subsection{b)}
    Welche funktionale Darstellung haben die Elektronenwellenfunktionen in einem Kristall nach dem Bloch'schen Theorem?
    Zeichnen und beschriften Sie eine Dispersionsrelation für Kristallelektronen (zwei Bänder).
    GEben Sie auch die daraus resultierende Dispersion der Gruppengeschwindigkeit des Elektrons an.

    % \subsection{c)}
    % fehlt!?

    \subsection{d)}
    Was besagen die Born-von-Kármán-Randbedingungen?

    \subsection{e)}
    Wie ist die effektive Masse definiert?
    Geben Sie den Hamiltonoperator für ein Kristallelektron vor und nach Einführung der effektiven Masse in der Einteilchennäherung an.
    Diese soll dafür skalar angenommen werden.

    \subsection{f1)} % im Scan doppelt f vergeben
    Wie ist die Zustandsdichte definiert?
    Wie lauten die funktionale Abhängigkeit der Zustandsdichte von der Energie für freie Elektronengase in $n$ Dimensionen ($N = 1, 2, 3$)?

    \subsection{f2)} % im Scan doppelt f vergeben
    Welche Statistik befolgen Phononen, welche befolgen Elektronen?
    Geben Sie die Verteilungsfunktionen An.
    Wie lässt sich das chemische Potential anschaulich erklären?

    \subsection{g)}
    Diskutieren Sie die Temperaturabhängigkeit der Wärmekapazität eines Kristalls aufgrund seiner Gitterschwingungen.
    Wie lautet die Dulong-Petit-Regel?
    Was ist das Einstein-, was das Debye-Modell?
    Und für welche Phononen sind dies gute Näherungen?
\end{aufgabe}

\begin{aufgabe}{Aufgabe 3: Wärmekapazität}
    Zeigen Sie, dass in Debyescher Näherung
    die Wärmekapazität eines zweidimensionales Gitters identischer Atome
    für tiefe Temperaturen ($T \ll \Theta_D$) proportional zu $(\sfrac{T}{\Theta_D})^2$ ist.
\end{aufgabe}

\begin{aufgabe}{Aufgabe 4: Fermikugel}
    Für welche Dichte $n$ eines freien Elektronengases erreicht in einem Metall mit kubisch flächenzentriertem Gitter mit Gitterkonstante $a$ die Fermi-Kugel gerade den Rand der ersten Brillouinzone?

    \textbf{Hinweise}
    \begin{itemize}
        \item Geben Sie die primitiven Gittervektoren eines fcc-Gitters an.
              Geben Sie die zugehörigen Vektoren des reziproken Gitters an.
        \item Bestimmen Sie damit den Radius der Fermikugel $k_F$.
        \item Geben Sie $k_F$ als Funktion der Teilchendichte $n$ an und vergleichen Sie.
    \end{itemize}
\end{aufgabe}

\end{document}
