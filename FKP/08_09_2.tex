\documentclass[
  bibliography=totoc,     % Literatur im Inhaltsverzeichnis
  captions=tableheading,  % Tabellenüberschriften
  titlepage=firstiscover, % Titelseite ist Deckblatt
]{scrartcl}

% Paket float verbessern
\usepackage{scrhack}

% Warnung, falls nochmal kompiliert werden muss
\usepackage[aux]{rerunfilecheck}

% unverzichtbare Mathe-Befehle
\usepackage{amsmath}
% viele Mathe-Symbole
\usepackage{amssymb}
% Erweiterungen für amsmath
\usepackage{mathtools}

% Fonteinstellungen
\usepackage{fontspec}
% Latin Modern Fonts werden automatisch geladen
% Alternativ zum Beispiel:
%\setromanfont{Libertinus Serif}
%\setsansfont{Libertinus Sans}
%\setmonofont{Libertinus Mono}

% Wenn man andere Schriftarten gesetzt hat,
% sollte man das Seiten-Layout neu berechnen lassen
\recalctypearea{}

% deutsche Spracheinstellungen
\usepackage[ngerman]{babel}


\usepackage[
  math-style=ISO,    % ┐
  bold-style=ISO,    % │
  sans-style=italic, % │ ISO-Standard folgen
  nabla=upright,     % │
  partial=upright,   % ┘
  warnings-off={           % ┐
    mathtools-colon,       % │ unnötige Warnungen ausschalten
    mathtools-overbracket, % │
  },                       % ┘
]{unicode-math}

% traditionelle Fonts für Mathematik
\setmathfont{Latin Modern Math}
% Alternativ zum Beispiel:
%\setmathfont{Libertinus Math}

% \setmathfont{XITS Math}[range={scr, bfscr}]
% \setmathfont{XITS Math}[range={cal, bfcal}, StylisticSet=1]

% Zahlen und Einheiten
\usepackage[
  locale=DE,                   % deutsche Einstellungen
  separate-uncertainty=true,   % immer Unsicherheit mit \pm
  per-mode=symbol-or-fraction, % / in inline math, fraction in display math
]{siunitx}

% chemische Formeln
\usepackage[
  version=4,
  math-greek=default, % ┐ mit unicode-math zusammenarbeiten
  text-greek=default, % ┘
]{mhchem}

% richtige Anführungszeichen
\usepackage[autostyle]{csquotes}

% schöne Brüche im Text
\usepackage{xfrac}

% Standardplatzierung für Floats einstellen
\usepackage{float}
\floatplacement{figure}{htbp}
\floatplacement{table}{htbp}

% Floats innerhalb einer Section halten
\usepackage[
  section, % Floats innerhalb der Section halten
  below,   % unterhalb der Section aber auf der selben Seite ist ok
]{placeins}

% Seite drehen für breite Tabellen: landscape Umgebung
\usepackage{pdflscape}

% Captions schöner machen.
\usepackage[
  labelfont=bf,        % Tabelle x: Abbildung y: ist jetzt fett
  font=small,          % Schrift etwas kleiner als Dokument
  width=0.9\textwidth, % maximale Breite einer Caption schmaler
]{caption}
% subfigure, subtable, subref
\usepackage{subcaption}

% Grafiken können eingebunden werden
\usepackage{graphicx}

% schöne Tabellen
\usepackage{booktabs}

% Verbesserungen am Schriftbild
\usepackage{microtype}

% Literaturverzeichnis
\usepackage[
  backend=biber,
]{biblatex}
% Quellendatenbank
\addbibresource{lit.bib}
% \addbibresource{programme.bib}

% Hyperlinks im Dokument
\usepackage[
  german,
  unicode,        % Unicode in PDF-Attributen erlauben
  pdfusetitle,    % Titel, Autoren und Datum als PDF-Attribute
  pdfcreator={},  % ┐ PDF-Attribute säubern
  pdfproducer={}, % ┘
]{hyperref}
% erweiterte Bookmarks im PDF
\usepackage{bookmark}

% Trennung von Wörtern mit Strichen
\usepackage[shortcuts]{extdash}

\usepackage{parskip}

\author{%
  Nicolai Weitkemper\\%
  \href{mailto:nicolai.weitkemper@tu-dortmund.de}{nicolai.weitkemper@tu-dortmund.de}%
}
\publishers{TU Dortmund – Fakultät Physik}

\global\def\colvec#1{\begin{pmatrix}#1\end{pmatrix}}
% \global\def\b#1{\left(#1\right)}
% \global\def\pdv#1{\frac{\partial}{\partial #1}}
\global\def\dv#1#2{\frac{\d #1}{\d #2}}
\global\def\pdv#1#2{\frac{\partial #1}{\partial #2}}
\global\def\pdvfest#1#2#3{\left.\frac{\partial #1}{\partial #2}\right\rvert_{#3}}
\global\def\d{\mathrm{d}}
\global\def\intfty{\int_{-\infty}^\infty} % ha ha
\global\def\half{\frac{1}{2}}
\global\def\quarter{\frac{1}{4}}

\global\def\b#1{\left(#1\right)}
\global\def\L{\mathcal{L}}
\global\def\normtwo{\frac{1}{\sqrt{2}}}


\global\def\kB{k_\text{B}}
\global\def\b#1{\left(#1\right)}
\global\def\a#1{\left\langle#1\right\rangle}

% https://tex.stackexchange.com/a/444226
\makeatletter
\renewcommand{\@seccntformat}[1]{}
\makeatother

\NewCommandCopy{\oldDelta}{\Delta}
\renewcommand{\Delta}{\mathrm{\oldDelta}}

\newenvironment{aufgabe}[1]
    {
    \section{#1}
    }
    {
    \clearpage
    }

% TODO: Nicht die neueste Version des Standard-header_common!


\subject{FKP}
\title{Festkörperphysik}
\date{
    Wintersemester 2008/2009
}

\begin{document}

\maketitle
\thispagestyle{empty}
\newpage


\begin{aufgabe}{Aufgabe 1: Kurzfragen}
    \subsection{a)}
    Zeichnen und beschriften Sie eine typische Dispersionsrelation für Phononen.
    Nehmen Sie dazu einen Kristall mit zweiatomiger Einheitszelle an.
    Warum kann man sich auf die erste Brillouinzone beschränken?

    \subsection{b)}
    Welche funktionale Darstellung haben die Elektronenwellenfunktionen in einem Kristall nach dem Bloch'schen Theorem?
    Zeichnen und beschriften Sie eine Dispersionsrelation für Kristallelektronen (zwei Bänder).
    GEben Sie auch die daraus resultierende Dispersion der Gruppengeschwindigkeit des Elektrons an.

    % \subsection{c)}
    % fehlt!?

    \subsection{d)}
    Was besagen die Born-von-Kármán-Randbedingungen?

    \subsection{e)}
    Wie ist die effektive Masse definiert?
    Geben Sie den Hamiltonoperator für ein Kristallelektron vor und nach Einführung der effektiven Masse in der Einteilchennäherung an.
    Diese soll dafür skalar angenommen werden.

    \subsection{f1)} % im Scan doppelt f vergeben
    Wie ist die Zustandsdichte definiert?
    Wie lauten die funktionale Abhängigkeit der Zustandsdichte von der Energie für freie Elektronengase in $n$ Dimensionen ($N = 1, 2, 3$)?

    \subsection{f2)} % im Scan doppelt f vergeben
    Welche Statistik befolgen Phononen, welche befolgen Elektronen?
    Geben Sie die Verteilungsfunktionen An.
    Wie lässt sich das chemische Potential anschaulich erklären?

    \subsection{g)}
    Diskutieren Sie die Temperaturabhängigkeit der Wärmekapazität eines Kristalls aufgrund seiner Gitterschwingungen.
    Wie lautet die Dulong-Petit-Regel?
    Was ist das Einstein-, was das Debye-Modell?
    Und für welche Phononen sind dies gute Näherungen?
\end{aufgabe}

\begin{aufgabe}{Aufgabe 3: Wärmekapazität}
    Zeigen Sie, dass in Debyescher Näherung
    die Wärmekapazität eines zweidimensionales Gitters identischer Atome
    für tiefe Temperaturen ($T \ll \Theta_D$) proportional zu $(\sfrac{T}{\Theta_D})^2$ ist.
\end{aufgabe}

\begin{aufgabe}{Aufgabe 4: Fermikugel}
    Für welche Dichte $n$ eines freien Elektronengases erreicht in einem Metall mit kubisch flächenzentriertem Gitter mit Gitterkonstante $a$ die Fermi-Kugel gerade den Rand der ersten Brillouinzone?

    \textbf{Hinweise}
    \begin{itemize}
        \item Geben Sie die primitiven Gittervektoren eines fcc-Gitters an.
              Geben Sie die zugehörigen Vektoren des reziproken Gitters an.
        \item Bestimmen Sie damit den Radius der Fermikugel $k_F$.
        \item Geben Sie $k_F$ als Funktion der Teilchendichte $n$ an und vergleichen Sie.
    \end{itemize}
\end{aufgabe}

\end{document}
