\input{header_common.tex}

\subject{FKP}
\title{Festkörperphysik}
\date{
    Wintersemester 2008/2009
}

\begin{document}

\maketitle
\thispagestyle{empty}
\newpage


\begin{aufgabe}{Aufgabe 1: Kurzfragen}
    % Fragen zum Kristallgitter
    \subsection{a)}
    In welche zwei Komponenten lässt sich ein Kristallgitter zerlegen?

    \subsection{b)}
    Zeichnen Sie eine primitive und eine nicht-primitive Einheitszelle in ein zweidimensionales quadratisches Gitter ein!
    Was ist die Wigner-Seitz-Zelle?

    \subsection{c)}
    Wie lautet die Definition des reziproken Gitters?
    Wie stehen die Vektoren des direkten und des reziproken Gitters zueinander?
    Was ist das reziproke Gitter des reziproken Gitters?

    \subsection{d)}
    Wie ist die erste Brillouinzone definiert?
    Es reicht die Beispielzeichnung eines eindimensionalen Gitters.

    % Fragen zur Strukturanalyse
    \subsection{e)}
    Wie lautet das Auswahlkriterium für das Auftreten von Röntgenreflexen,
    das rein aus der Geometrie des Kristallgitters folgt?

    \subsection{f)}
    Überführen Sie diese Auswahlregel in die Laue-Gleichungen!

    \subsection{g)}
    Was ist der Strukturfaktor bzw. der Atomformfaktor?

    \subsection{h)}
    Welche Strahlungstypen könnte man für die Strukturanalyse verwenden und welche Wellenlängen müssen verwendet werden?

    \subsection{i)}
    Wenn Sie auf den Kristall mit monochromatischer Röntgenstrahlung einstrahlen,
    finden Sie in der Regel keinen Reflex.
    Wie müssen Sie Ihre experimentellen Möglichkeiten erweitern,
    um Reflexe beobachten zu können?

    % Sonstiges
    \subsection{j)}
    Was ist die Grundidee der Born-Oppenheimer-Näherung?
\end{aufgabe}
\end{document}
