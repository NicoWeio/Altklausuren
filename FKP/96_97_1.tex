\input{header_common.tex}

\subject{FKP}
\title{Festkörperphysik}
\date{
    Wintersemester 1996/1997
}

\begin{document}

\maketitle
\thispagestyle{empty}
\newpage


\begin{aufgabe}{Aufgabe 1}
    Die Verbindung \ce{Cu3Au} kristallisiert in einer fcc-Struktur.
    Dabei können die \ce{Au}- und die \ce{Cu}-Atome die Gitterplätze entweder statistisch besetzen oder aber geordnet vorliegen.

    \subsection{a)}
    Skizzieren Sie die Einheitszelle der geordneten Struktur.

    \subsection{b)}
    Berechnen Sie den Strukturfaktor $S_1(\vec K)$ unter der Annahme,
    dass die \ce{Au}- und die \ce{Cu}-Atome \textit{geordnet} vorliegen.
    Dabei sitze \ce{Au} im Ursprung der kubischen Elementarzelle des fcc-Gitters.

    \subsection{c)}
    Berechnen Sie den Strukturfaktor $S_2(\vec K)$ unter der Annahme,
    dass die \ce{Au}- und die \ce{Cu}-Atome \textit{statistisch} verteilt sind.

    \subsection{d)}
    Mittels welchen Experiments kann man zwischen der geordneten und der statistischen Anordnung unterscheiden?
    (Die Kernladungszahlen sind $Z_{\ce{Au}} = 79$ und $Z_{\ce{Cu}} = 29$.)
\end{aufgabe}

\begin{aufgabe}{Aufgabe 7}
    Ein thermisches Neutron mit der Primärenergie $E_0$ und dem Impuls $\hbar \vec k_0$ übertrage
    bei einem Rückstreuexperiment (Streuwinkel $\SI{180}{\degree}$) bzw.
    bei einem Vorwärtsstreuexperiment (Streuwinkel $\SI{0}{\degree}$)
    die Energie $\sfrac{E_0}{2}$ auf das streuende System,
    das volle Translationssymmetrie besitzen soll.
    Berechnen Sie für die beiden Fälle den \textit{Vektor} $\hbar \vec q$ des Impulsübertrages.
\end{aufgabe}
\end{document}
