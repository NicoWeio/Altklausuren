\documentclass[
  bibliography=totoc,     % Literatur im Inhaltsverzeichnis
  captions=tableheading,  % Tabellenüberschriften
  titlepage=firstiscover, % Titelseite ist Deckblatt
]{scrartcl}

% Paket float verbessern
\usepackage{scrhack}

% Warnung, falls nochmal kompiliert werden muss
\usepackage[aux]{rerunfilecheck}

% unverzichtbare Mathe-Befehle
\usepackage{amsmath}
% viele Mathe-Symbole
\usepackage{amssymb}
% Erweiterungen für amsmath
\usepackage{mathtools}

% Fonteinstellungen
\usepackage{fontspec}
% Latin Modern Fonts werden automatisch geladen
% Alternativ zum Beispiel:
%\setromanfont{Libertinus Serif}
%\setsansfont{Libertinus Sans}
%\setmonofont{Libertinus Mono}

% Wenn man andere Schriftarten gesetzt hat,
% sollte man das Seiten-Layout neu berechnen lassen
\recalctypearea{}

% deutsche Spracheinstellungen
\usepackage[ngerman]{babel}


\usepackage[
  math-style=ISO,    % ┐
  bold-style=ISO,    % │
  sans-style=italic, % │ ISO-Standard folgen
  nabla=upright,     % │
  partial=upright,   % ┘
  warnings-off={           % ┐
    mathtools-colon,       % │ unnötige Warnungen ausschalten
    mathtools-overbracket, % │
  },                       % ┘
]{unicode-math}

% traditionelle Fonts für Mathematik
\setmathfont{Latin Modern Math}
% Alternativ zum Beispiel:
%\setmathfont{Libertinus Math}

% \setmathfont{XITS Math}[range={scr, bfscr}]
% \setmathfont{XITS Math}[range={cal, bfcal}, StylisticSet=1]

% Zahlen und Einheiten
\usepackage[
  locale=DE,                   % deutsche Einstellungen
  separate-uncertainty=true,   % immer Unsicherheit mit \pm
  per-mode=symbol-or-fraction, % / in inline math, fraction in display math
]{siunitx}

% chemische Formeln
\usepackage[
  version=4,
  math-greek=default, % ┐ mit unicode-math zusammenarbeiten
  text-greek=default, % ┘
]{mhchem}

% richtige Anführungszeichen
\usepackage[autostyle]{csquotes}

% schöne Brüche im Text
\usepackage{xfrac}

% Standardplatzierung für Floats einstellen
\usepackage{float}
\floatplacement{figure}{htbp}
\floatplacement{table}{htbp}

% Floats innerhalb einer Section halten
\usepackage[
  section, % Floats innerhalb der Section halten
  below,   % unterhalb der Section aber auf der selben Seite ist ok
]{placeins}

% Seite drehen für breite Tabellen: landscape Umgebung
\usepackage{pdflscape}

% Captions schöner machen.
\usepackage[
  labelfont=bf,        % Tabelle x: Abbildung y: ist jetzt fett
  font=small,          % Schrift etwas kleiner als Dokument
  width=0.9\textwidth, % maximale Breite einer Caption schmaler
]{caption}
% subfigure, subtable, subref
\usepackage{subcaption}

% Grafiken können eingebunden werden
\usepackage{graphicx}

% schöne Tabellen
\usepackage{booktabs}

% Verbesserungen am Schriftbild
\usepackage{microtype}

% Literaturverzeichnis
\usepackage[
  backend=biber,
]{biblatex}
% Quellendatenbank
\addbibresource{lit.bib}
% \addbibresource{programme.bib}

% Hyperlinks im Dokument
\usepackage[
  german,
  unicode,        % Unicode in PDF-Attributen erlauben
  pdfusetitle,    % Titel, Autoren und Datum als PDF-Attribute
  pdfcreator={},  % ┐ PDF-Attribute säubern
  pdfproducer={}, % ┘
]{hyperref}
% erweiterte Bookmarks im PDF
\usepackage{bookmark}

% Trennung von Wörtern mit Strichen
\usepackage[shortcuts]{extdash}

\usepackage{parskip}

\author{%
  Nicolai Weitkemper\\%
  \href{mailto:nicolai.weitkemper@tu-dortmund.de}{nicolai.weitkemper@tu-dortmund.de}%
}
\publishers{TU Dortmund – Fakultät Physik}

\global\def\colvec#1{\begin{pmatrix}#1\end{pmatrix}}
% \global\def\b#1{\left(#1\right)}
% \global\def\pdv#1{\frac{\partial}{\partial #1}}
\global\def\dv#1#2{\frac{\d #1}{\d #2}}
\global\def\pdv#1#2{\frac{\partial #1}{\partial #2}}
\global\def\pdvfest#1#2#3{\left.\frac{\partial #1}{\partial #2}\right\rvert_{#3}}
\global\def\d{\mathrm{d}}
\global\def\intfty{\int_{-\infty}^\infty} % ha ha
\global\def\half{\frac{1}{2}}
\global\def\quarter{\frac{1}{4}}

\global\def\b#1{\left(#1\right)}
\global\def\L{\mathcal{L}}
\global\def\normtwo{\frac{1}{\sqrt{2}}}


\global\def\kB{k_\text{B}}
\global\def\b#1{\left(#1\right)}
\global\def\a#1{\left\langle#1\right\rangle}

% https://tex.stackexchange.com/a/444226
\makeatletter
\renewcommand{\@seccntformat}[1]{}
\makeatother

\NewCommandCopy{\oldDelta}{\Delta}
\renewcommand{\Delta}{\mathrm{\oldDelta}}

\newenvironment{aufgabe}[1]
    {
    \section{#1}
    }
    {
    \clearpage
    }

% TODO: Nicht die neueste Version des Standard-header_common!


\subject{KET}
\title{Kern- und Elementarteilchenphysik}
\date{
    Wintersemester 2007/2008
}

\begin{document}

\maketitle
\thispagestyle{empty}
\newpage


\begin{aufgabe}{Aufgabe 1: Kurzfragen}
    \subsection{1.}
    Das $K^{*0}$ zerfällt dominant in ein geladenes Kaon und ein geladenes Pion.
    Welche Ladung trägt das Pion?

    \subsection{2.}
    Die Reichweite von starker und schwacher Wechselwirkung ist begrenzt.
    Was ist der Unterschied in den zugrundeliegenden Mechanismen?

    \subsection{3.}
    In hochenergetischen Elektron-Proton-Streureaktionen beobachten Sie,
    dass das Elektron häufig an einem Antiquark im Proton streut.
    Interpretieren Sie kurz die Beobachtung!

    \subsection{4.}
    Der Zerfall $\ce{^22_11Na} \to Xe^+\nu_e$ wird beobachtet,
    der Zerfall $\ce{^1_1H} \to Xe^+\nu_e$
    ($X$ sei der jeweilige Tochterkern)
    hingegen nicht.
    Warum?

    \subsection{5.}
    Ein Tritiumkern und ein Wasserstoffkern werden auf eine Energie von jeweils $\SI{5}{\giga\electronvolt}$ beschleunigt
    und in ein transparentes Medium mit einem Brechungsindex von $n = \num{1.5}$ geschossen.
    Sie beobachten, dass bezüglich der Flugbahnen der Teilchen Licht emittiert wird.
    Für welches Teilchen ist der Winkel, unter dem das Licht emittiert wird, größer?
    Begründen Sie Ihre Antwort!

    \subsection{6.}
    Betrachten Sie zwei Isobare mit unterschiedlichen Kernladungszahlen $Z_2 > Z_1$ und Lebensdauern $\tau_2$ und $\tau_1$.
    Beide seinen instabil gegen $\alpha$-Zerfälle.
    Geben Sie einen Grund dafür an, dass $\tau_2 > \tau_1$ ist!

    \subsection{7.}
    Was ist die Bedeutung eines Formfaktors in Streureaktionen der Kern- und Teilchenphysik?

    \subsection{8.}
    Kernkräfte werden auch durch den Austausch von Pionen beschrieben.
    Schätzen Sie anhand der Pionmasse ($m_\pi \simeq \SI{140}{\mega\electronvolt\per{c}\squared}$) ihre Reichweite ab!

    \subsection{9.}
    Die Existenz des $\Omega^-$ lässt sich nur durch die Existenz eines Farbfreiheitsgrades erklären.
    Warum?

    % \subsection{10.}

    \subsection{11.}
    Bis zu welcher ungefähren Masse können Kerne in Fusionsreaktionen in Sternen erzeugt werden?
    Begründen Sie Ihre Antwort!

    \subsection{12.}
    Was versteht man unter \enquote{magischen Zahlen}?
\end{aufgabe}
\end{document}
