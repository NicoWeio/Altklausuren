\input{header_common.tex}

\subject{KET}
\title{Kern- und Elementarteilchenphysik}
\date{
    Wintersemester 1998/1999
}

\begin{document}

\maketitle
\thispagestyle{empty}
\newpage


\begin{aufgabe}{Aufgabe 1: Kurzfragen}
    % \subsection{1.}
    % \subsection{2.}
    % \subsection{3.}
    % \subsection{4.}

    \subsection{5.}
    Welche Phänomene sind dafür verantwortlich, dass bei der Kernspaltung schwerer Kerne Energie frei wird?

    \subsection{6.}
    Nennen Sie mindestens 2 experimentelle Ergebnisse, die zeigen,
    dass neben dem $e^-$ beim $\beta$-Zerfall ein $\bar\nu$ emittiert wird.

    \subsection{7.}
    Nennen Sie mindestens drei physikalische Eigenschaften der Kernspaltung,
    die notwendig sind, damit Kernreaktoren arbeiten und genutzt werden können.

    \subsection{8.}
    Welche Reaktionen sind für die Energieproduktion in der Sonne verantwortlich?
    Welche Teilreaktion legt den zeitlichen Ablauf fest?

    \subsection{9.}
    Wie entstehen schwere Elemente $Z > 30$ in Sternen?

    \subsection{10.}
    Wie zeigt man, dass die elektromagnetische und starke Wechselwirkung invariant gegen Zeitumkehr sind?

    \subsection{11.}
    Welche experimentelle Evidenz existiert dafür, dass Quarks Spin ½ besitzen (≥ 2 Methoden)?

    \subsection{12.}
    Wie kann man die Farbe (colour) der Quarks experimentell bestimmen?
    Geben Sie mindestens 2 Methoden an.

    \subsection{13.}
    Welche Information gewinnt man aus der Analyse der Quarkonium-Zustände?

    \subsection{14.}
    Geben Sie mindestens ein Argument dafür an,
    dass die Resultate der tiefinelastischen Lepton-Proton Streuung
    auf die Elektron-Quark-Streuung zurückgeführt werden können.

    \subsection{15.}
    Welcher Multipol dominiert bei dem Übergang
    \begin{align*}
        % TODO: suboptimal gesetzt
        %
        % 1 \ce{^3S_1} &\to 1 \ce{^3S_0} + \gamma \\
        % 2 \ce{^3S_1} &\to 1 \ce{^3P_2} + \gamma
        %
        1 \; {}^3S_1 &\to 1 \; {}^3S_0 + \gamma \\
        2 \; {}^3S_1 &\to 1 \; {}^3P_2 + \gamma
    \end{align*}
    des Charmoniums?
    Geben Sie die Auswahlregel an.

\end{aufgabe}
\end{document}
