\input{header_common.tex}

\subject{KET}
\title{Kern- und Elementarteilchenphysik}
\date{
    Wintersemester 2009/2010
}

\begin{document}

\maketitle
\thispagestyle{empty}
\newpage


\begin{aufgabe}{Aufgabe 1: Kurzfragen}
    \subsection{1.)}
    Ein Teilchen $X$ habe zwei Zerfallskanäle mit partiellen Zerfallsbreiten $\Gamma_1$ und $\Gamma_2$.
    Wie groß sind mittlere Lebensdauer und Gesamtbreite von $X$?

    % \subsection{2.)}
    % In untenstehender Abbildung ist der Energieverlust $-\dv{E}{x}$ gegen den Teilchenimpuls für verschiedene Teilchenarten aufgetragen.
    % Identifizieren Sie Proton, Elektron, Pion und Deuteron.
    % Bei welchem Impuls bezeichnet man das Kaon als \enquote{MIP}?
    %
    % TODO: Abbildung

    % \subsection{3.)}
    % Gegeben sei der folgende Querschnitt durch einen Detektor, in dem ein homogenes $B$-Feld in positiver $z$-Richtung vorliegt.
    % Zeichnen Sie für die jeweiligen Teilchenarten typische Signaturen (gemessene Signale) in den einzelnen Detektorkomponenten ein.

    % Stellen Sie die Bahnen von Teilchen, die in einem System keine Signaturen hinterlassen, als gestrichelte Linie in diesem System dar.
    % Nehmen Sie an, dass alle Teilchen den gleichen Impuls innehaben und dass dieser Impuls ausreiche, um das Magnetfeld zu verlassen.

    % Der Detektor bestehe aus Spurkammer (SK), elektromagnetischem Kalorimeter (EMK), hadronischem Kalorimeter (HK) und Myonsystem (MS).
    %
    % TODO: Abbildung

    \subsection{4.)}
    Welcher physikalische Effekt ermöglicht es, Kernreaktoren wie Druck- und Siedewasser-Reaktoren aktiv zu regeln?

    \subsection{5.)}
    Reine Neutronenkerne mit mehr als einem Neutron werden in der Natur nicht beobachtet.
    Erklären Sie, warum nicht.

    \subsection{6.)}
    \ce{^48_20Ca} befindet sich außerhalb des \textit{Tals der Stabilität}.
    Geben Sie einen möglichen Grund an, weswegen es trotzdem nahezu stabil ist ($\tau = \SI{4.3E19}{\year}$).

    \subsection{7.)}
    Sind folgende Teilchen im Quarkmodell realisiert?
    Begründen Sie Ihre Antwort.

    \subsubsection{a)} Antibaryon mit elektrischer Ladung $Q / |e| = +2$.
    \subsubsection{b)} Meson mit elektrischer Ladung $Q / |e| = +1$ und Strangeness $-1$.
    \subsubsection{c)} Meson mit $J^{PC} = 0^{-+}$.

    \subsection{8.)}
    Die leichtesten Charmmesonen sind $D^0$, $\bar D^0$ und $D^\pm$.
    Welchen Quarkinhalt haben diese Mesonen?
    Begründen Sie, warum diese $D$-Mesonen nur schwach zerfallen können.

    \subsection{9.)}
    Ein hypothetisches Teilchen zerfalle sowohl in zwei als auch in drei Pionen.
    Es werde kein relativer Bahndrehimpuls zwischen den Pionen beobachtet.
    Über welche Wechselwirkung fänden die Zerfälle statt?
    Begründen Sie Ihre Antwort kurz.
\end{aufgabe}
\end{document}
