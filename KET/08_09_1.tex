\input{header_common.tex}

\subject{KET}
\title{Kern- und Elementarteilchenphysik}
\date{
    Wintersemester 2008/2009
}

\begin{document}

\maketitle
\thispagestyle{empty}
\newpage


\begin{aufgabe}{Aufgabe 1: Kurzfragen}
    \subsection{(1)}
    Wie groß ist die typische Bindungsenergie pro Nukleon in mittelschweren Kernen?

    \subsection{(2)}
    Welchen Radius hat ungefähr
    \begin{enumerate}[label=(\roman*)]
        \item ein Proton?
        \item ein Elektron?
        \item ein Urankern?
    \end{enumerate}

    \subsection{(3)}
    Was ist der Unterschied zwischen $\beta^+$ und EC-Zerfällen?

    \subsection{(4)}
    Was ist die \enquote{Spaltbarriere}?

    \subsection{(5)}
    Was hat Kernspaltung mit dem Tunneleffekt zu tun?

    \subsection{(6)}
    Was versteht man unter \enquote{verzögerten Neutronen} und wofür sind sie wichtig?

    \subsection{(7)}
    Was bedeuten $\ce{^235U}(n,f)$ und $\ce{^238U}(n,n')$?

    \subsection{(8)}
    Was beinhaltet die 4-Faktor-Formel?

    \subsection{(9)}
    Welches sind die Moderatoren eines Schwerwasserreaktors und eines RBMK?

    \subsection{(10)}
    Was unterscheidet den \enquote{Schnellen Brüter} von einem \enquote{Thorium-Brüter}?

    \subsection{(11)}
    Was unterscheidet die \enquote{Äquivalentdosis} von der \enquote{Energiedosis}?

    \subsection{(12)}
    Wie rechnet man \enquote{\unit{{rad}}} in \enquote{\unit{{Gy}}} um?

    \subsection{(13)}
    Ab welcher Ganzkörper-Strahlendosis muss mit etwa \SI{50}{\percent} Todesfolge gerechnet werden?

    \subsection{(14)}
    Welche ungefähre kinetische Energie haben thermische Neutronen?

    \subsection{(15)}
    Wie kann man verstehen, dass das magnetische Moment des neutralen Neutrons von Null verschieden ist?

    \subsection{(16)}
    Was ist der dominante Energieverlust für schwere geladene Teilchen in Materie?

    \subsection{(17)}
    Welcher Wechselwirkungsprozess ist für hochenergetische Photonen in Materie dominant?

    \subsection{(18)}
    Mit welchem Detektor kann man die Geschwindigkeit eines Teilchens bestimmen?

    \subsection{(19)}
    Was ist ein \enquote{mip}?

    \subsection{(20)}
    Skizzieren Sie den schematischen Aufbau eines modernen Detektors an einem Kollider
    und benennen Sie die wichtigsten Komponenten.

    \subsection{(21)}
    Reihen Sie die Ihnen bekannten Quark-Flavours in sinnvolle Gruppen ein.

    \subsection{(22)}
    An welche Händigkeit der Teilchen koppelt die schwache Wechselwirkung?

    \subsection{(23)}
    Was lässt die schwache Wechselwirkung so schwach erscheinen?

    \subsection{(24)}
    Skizzieren Sie das Multiplett der Spin-½-Baryonen der $\operatorname{SU}(3)_\text{flavour}$.
    Denken Sie auch an eine sinnvolle Achsenbeschriftung.

    \subsection{(25)}
    Nennen Sie jeweils mindestens einen Grund, warum die folgenden Reaktionen nicht möglich sein sollten:

    \begin{enumerate}[label=(\roman*)]
        \item $p + p \to p + p + n + \bar\Lambda$
        \item $e^+ + \mu^- \to e^- + \mu^+$
        \item $p \to n + e^+ + \nu_e$
        \item $K^+ \to \pi^+ + \gamma$
    \end{enumerate}

    \subsection{(26)}
    Welche See-Quarks finden sich im Neutron?

    \subsection{(27)}
    In welchen Isospinzuständen kann man Pionen antreffen?

    \subsection{(28)}
    Skizzieren Sie ein Feynman-Diagramm für den Betazerfall des Neutrons auf Quarkniveau.

    \subsection{(29)}
    Was ist der Unterschied zwischen einem reellen und einem virtuellen Teilchen?

    \subsection{(30)}
    Welche Ruhemasse hat ein Photon?

    \subsection{(31)}
    Was kann man aus der Breite des $Z^0$ ablesen?

    \subsection{(32)}
    Welchen Vorteil kann ein Experiment an einem Kollider gegenüber einem Fixed-Target-Experiment nutzen?

    \subsection{(33)}
    Welches sind die Ziele des Physikprogramms am Large Hadron Collider?
    (2 Ziele genügen.)

    \subsection{(34)}
    Was ist die ungefähre Lebensdauer freier Protonen?

    \subsection{(35)}
    Welche ungefähre Masse hat ein $W$-Boson?

    \subsection{(36)}
    Warum ist der Zerfall $\pi^+ \to \mu^+ + \nu_\mu$ deutlich häufiger als der konkurrierende Zerfall $\pi^+ \to e^+ + \nu_e$?

    \subsection{(37)}
    Was versteht man unter einer CP-Transformation? Geben Sie ein Beispiel.

    \subsection{(38)}
    Mit welchem historischen Experiment konnte die Helizität der Neutrinos bestimmt werden?

    \subsection{(39)}
    Was ist die Fouriertransformierte einer Exponential-Verteilung?
    In welchem Zusammenhang der Kern- und Teilchenphysik ist das von Bedeutung?

    \subsection{(40)}
    Wie lauten die Antiteilchen zum $K^0$ und zum $\pi^0$?

    \subsection{(41)}
    Wie lautet das Antiteilchen zum Neutron?

    \subsection{(42)}
    Sie entdecken nach ausgiebigem Experimentieren und Analyse Ihrer Ergebnisse ein Indiz für die Verletzung der CPT-Invarianz.
    Welche Kapitel für eine Neuauflage eines herauszubringenden Lehrbuchs über Teilchenphysik müssten bei Bestätigung Ihrer Resultate neu geschrieben werden?
    Geben Sie dazu einige Stichworte.
\end{aufgabe}
\end{document}
