\documentclass[
  bibliography=totoc,     % Literatur im Inhaltsverzeichnis
  captions=tableheading,  % Tabellenüberschriften
  titlepage=firstiscover, % Titelseite ist Deckblatt
]{scrartcl}

% Paket float verbessern
\usepackage{scrhack}

% Warnung, falls nochmal kompiliert werden muss
\usepackage[aux]{rerunfilecheck}

% unverzichtbare Mathe-Befehle
\usepackage{amsmath}
% viele Mathe-Symbole
\usepackage{amssymb}
% Erweiterungen für amsmath
\usepackage{mathtools}

% Fonteinstellungen
\usepackage{fontspec}
% Latin Modern Fonts werden automatisch geladen
% Alternativ zum Beispiel:
%\setromanfont{Libertinus Serif}
%\setsansfont{Libertinus Sans}
%\setmonofont{Libertinus Mono}

% Wenn man andere Schriftarten gesetzt hat,
% sollte man das Seiten-Layout neu berechnen lassen
\recalctypearea{}

% deutsche Spracheinstellungen
\usepackage[ngerman]{babel}


\usepackage[
  math-style=ISO,    % ┐
  bold-style=ISO,    % │
  sans-style=italic, % │ ISO-Standard folgen
  nabla=upright,     % │
  partial=upright,   % ┘
  warnings-off={           % ┐
    mathtools-colon,       % │ unnötige Warnungen ausschalten
    mathtools-overbracket, % │
  },                       % ┘
]{unicode-math}

% traditionelle Fonts für Mathematik
\setmathfont{Latin Modern Math}
% Alternativ zum Beispiel:
%\setmathfont{Libertinus Math}

% \setmathfont{XITS Math}[range={scr, bfscr}]
% \setmathfont{XITS Math}[range={cal, bfcal}, StylisticSet=1]

% Zahlen und Einheiten
\usepackage[
  locale=DE,                   % deutsche Einstellungen
  separate-uncertainty=true,   % immer Unsicherheit mit \pm
  per-mode=symbol-or-fraction, % / in inline math, fraction in display math
]{siunitx}

% chemische Formeln
\usepackage[
  version=4,
  math-greek=default, % ┐ mit unicode-math zusammenarbeiten
  text-greek=default, % ┘
]{mhchem}

% richtige Anführungszeichen
\usepackage[autostyle]{csquotes}

% schöne Brüche im Text
\usepackage{xfrac}

% Standardplatzierung für Floats einstellen
\usepackage{float}
\floatplacement{figure}{htbp}
\floatplacement{table}{htbp}

% Floats innerhalb einer Section halten
\usepackage[
  section, % Floats innerhalb der Section halten
  below,   % unterhalb der Section aber auf der selben Seite ist ok
]{placeins}

% Seite drehen für breite Tabellen: landscape Umgebung
\usepackage{pdflscape}

% Captions schöner machen.
\usepackage[
  labelfont=bf,        % Tabelle x: Abbildung y: ist jetzt fett
  font=small,          % Schrift etwas kleiner als Dokument
  width=0.9\textwidth, % maximale Breite einer Caption schmaler
]{caption}
% subfigure, subtable, subref
\usepackage{subcaption}

% Grafiken können eingebunden werden
\usepackage{graphicx}

% schöne Tabellen
\usepackage{booktabs}

% Verbesserungen am Schriftbild
\usepackage{microtype}

% Literaturverzeichnis
\usepackage[
  backend=biber,
]{biblatex}
% Quellendatenbank
\addbibresource{lit.bib}
% \addbibresource{programme.bib}

% Hyperlinks im Dokument
\usepackage[
  german,
  unicode,        % Unicode in PDF-Attributen erlauben
  pdfusetitle,    % Titel, Autoren und Datum als PDF-Attribute
  pdfcreator={},  % ┐ PDF-Attribute säubern
  pdfproducer={}, % ┘
]{hyperref}
% erweiterte Bookmarks im PDF
\usepackage{bookmark}

% Trennung von Wörtern mit Strichen
\usepackage[shortcuts]{extdash}

\usepackage{parskip}

\author{%
  Nicolai Weitkemper\\%
  \href{mailto:nicolai.weitkemper@tu-dortmund.de}{nicolai.weitkemper@tu-dortmund.de}%
}
\publishers{TU Dortmund – Fakultät Physik}

\global\def\colvec#1{\begin{pmatrix}#1\end{pmatrix}}
% \global\def\b#1{\left(#1\right)}
% \global\def\pdv#1{\frac{\partial}{\partial #1}}
\global\def\dv#1#2{\frac{\d #1}{\d #2}}
\global\def\pdv#1#2{\frac{\partial #1}{\partial #2}}
\global\def\pdvfest#1#2#3{\left.\frac{\partial #1}{\partial #2}\right\rvert_{#3}}
\global\def\d{\mathrm{d}}
\global\def\intfty{\int_{-\infty}^\infty} % ha ha
\global\def\half{\frac{1}{2}}
\global\def\quarter{\frac{1}{4}}

\global\def\b#1{\left(#1\right)}
\global\def\L{\mathcal{L}}
\global\def\normtwo{\frac{1}{\sqrt{2}}}


\global\def\kB{k_\text{B}}
\global\def\b#1{\left(#1\right)}
\global\def\a#1{\left\langle#1\right\rangle}

% https://tex.stackexchange.com/a/444226
\makeatletter
\renewcommand{\@seccntformat}[1]{}
\makeatother

\NewCommandCopy{\oldDelta}{\Delta}
\renewcommand{\Delta}{\mathrm{\oldDelta}}

\newenvironment{aufgabe}[1]
    {
    \section{#1}
    }
    {
    \clearpage
    }

% TODO: Nicht die neueste Version des Standard-header_common!


\subject{KET}
\title{Kern- und Elementarteilchenphysik}
\date{
    Wintersemester 2008/2009
}

\begin{document}

\maketitle
\thispagestyle{empty}
\newpage


\begin{aufgabe}{Aufgabe 1: Kurzfragen}
    \subsection{(1)}
    Wie groß ist die typische Bindungsenergie pro Nukleon in mittelschweren Kernen?

    \subsection{(2)}
    Welchen Radius hat ungefähr
    \begin{enumerate}[label=(\roman*)]
        \item ein Proton?
        \item ein Elektron?
        \item ein Urankern?
    \end{enumerate}

    \subsection{(3)}
    Was ist der Unterschied zwischen $\beta^+$ und EC-Zerfällen?

    \subsection{(4)}
    Was ist die \enquote{Spaltbarriere}?

    \subsection{(5)}
    Was hat Kernspaltung mit dem Tunneleffekt zu tun?

    \subsection{(6)}
    Was versteht man unter \enquote{verzögerten Neutronen} und wofür sind sie wichtig?

    \subsection{(7)}
    Was bedeuten $\ce{^235U}(n,f)$ und $\ce{^238U}(n,n')$?

    \subsection{(8)}
    Was beinhaltet die 4-Faktor-Formel?

    \subsection{(9)}
    Welches sind die Moderatoren eines Schwerwasserreaktors und eines RBMK?

    \subsection{(10)}
    Was unterscheidet den \enquote{Schnellen Brüter} von einem \enquote{Thorium-Brüter}?

    \subsection{(11)}
    Was unterscheidet die \enquote{Äquivalentdosis} von der \enquote{Energiedosis}?

    \subsection{(12)}
    Wie rechnet man \enquote{\unit{{rad}}} in \enquote{\unit{{Gy}}} um?

    \subsection{(13)}
    Ab welcher Ganzkörper-Strahlendosis muss mit etwa \SI{50}{\percent} Todesfolge gerechnet werden?

    \subsection{(14)}
    Welche ungefähre kinetische Energie haben thermische Neutronen?

    \subsection{(15)}
    Wie kann man verstehen, dass das magnetische Moment des neutralen Neutrons von Null verschieden ist?

    \subsection{(16)}
    Was ist der dominante Energieverlust für schwere geladene Teilchen in Materie?

    \subsection{(17)}
    Welcher Wechselwirkungsprozess ist für hochenergetische Photonen in Materie dominant?

    \subsection{(18)}
    Mit welchem Detektor kann man die Geschwindigkeit eines Teilchens bestimmen?

    \subsection{(19)}
    Was ist ein \enquote{mip}?

    \subsection{(20)}
    Skizzieren Sie den schematischen Aufbau eines modernen Detektors an einem Kollider
    und benennen Sie die wichtigsten Komponenten.

    \subsection{(21)}
    Reihen Sie die Ihnen bekannten Quark-Flavours in sinnvolle Gruppen ein.

    \subsection{(22)}
    An welche Händigkeit der Teilchen koppelt die schwache Wechselwirkung?

    \subsection{(23)}
    Was lässt die schwache Wechselwirkung so schwach erscheinen?

    \subsection{(24)}
    Skizzieren Sie das Multiplett der Spin-½-Baryonen der $\operatorname{SU}(3)_\text{flavour}$.
    Denken Sie auch an eine sinnvolle Achsenbeschriftung.

    \subsection{(25)}
    Nennen Sie jeweils mindestens einen Grund, warum die folgenden Reaktionen nicht möglich sein sollten:

    \begin{enumerate}[label=(\roman*)]
        \item $p + p \to p + p + n + \bar\Lambda$
        \item $e^+ + \mu^- \to e^- + \mu^+$
        \item $p \to n + e^+ + \nu_e$
        \item $K^+ \to \pi^+ + \gamma$
    \end{enumerate}

    \subsection{(26)}
    Welche See-Quarks finden sich im Neutron?

    \subsection{(27)}
    In welchen Isospinzuständen kann man Pionen antreffen?

    \subsection{(28)}
    Skizzieren Sie ein Feynman-Diagramm für den Betazerfall des Neutrons auf Quarkniveau.

    \subsection{(29)}
    Was ist der Unterschied zwischen einem reellen und einem virtuellen Teilchen?

    \subsection{(30)}
    Welche Ruhemasse hat ein Photon?

    \subsection{(31)}
    Was kann man aus der Breite des $Z^0$ ablesen?

    \subsection{(32)}
    Welchen Vorteil kann ein Experiment an einem Kollider gegenüber einem Fixed-Target-Experiment nutzen?

    \subsection{(33)}
    Welches sind die Ziele des Physikprogramms am Large Hadron Collider?
    (2 Ziele genügen.)

    \subsection{(34)}
    Was ist die ungefähre Lebensdauer freier Protonen?

    \subsection{(35)}
    Welche ungefähre Masse hat ein $W$-Boson?

    \subsection{(36)}
    Warum ist der Zerfall $\pi^+ \to \mu^+ + \nu_\mu$ deutlich häufiger als der konkurrierende Zerfall $\pi^+ \to e^+ + \nu_e$?

    \subsection{(37)}
    Was versteht man unter einer CP-Transformation? Geben Sie ein Beispiel.

    \subsection{(38)}
    Mit welchem historischen Experiment konnte die Helizität der Neutrinos bestimmt werden?

    \subsection{(39)}
    Was ist die Fouriertransformierte einer Exponential-Verteilung?
    In welchem Zusammenhang der Kern- und Teilchenphysik ist das von Bedeutung?

    \subsection{(40)}
    Wie lauten die Antiteilchen zum $K^0$ und zum $\pi^0$?

    \subsection{(41)}
    Wie lautet das Antiteilchen zum Neutron?

    \subsection{(42)}
    Sie entdecken nach ausgiebigem Experimentieren und Analyse Ihrer Ergebnisse ein Indiz für die Verletzung der CPT-Invarianz.
    Welche Kapitel für eine Neuauflage eines herauszubringenden Lehrbuchs über Teilchenphysik müssten bei Bestätigung Ihrer Resultate neu geschrieben werden?
    Geben Sie dazu einige Stichworte.
\end{aufgabe}
\end{document}
