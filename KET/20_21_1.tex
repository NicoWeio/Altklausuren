\input{header_common.tex}

\subject{KET}
\title{Kern- und Elementarteilchenphysik}
\date{
    Wintersemester 2020/2021
}

\begin{document}

\maketitle
\thispagestyle{empty}
\newpage


\begin{aufgabe}{Aufgabe 1: Kurzfragen}
    \subsection{(a)}
    Die Existenz des $\Omega^-$ (Strangeness $S = -3$) lässt sich nur mithilfe eines Farbfreiheitsgrades erklären.
    Warum?

    \subsection{(b)}
    Welche Fusionsreaktion wird für zukünftige Fusionsreaktoren favorisiert?
    Begründen Sie Ihre Entscheidung.
    \subsection{(c)}
    Geben Sie für die folgenden Zerfälle und Reaktionen an, ob sie verboten oder erlaubt sind.
    Begründen Sie Ihre Antworten!
    Skizzieren Sie für die erlaubten Zerfälle jeweils ein Feynman-Diagramm auf Quarkniveau.
    \begin{enumerate}
        \item $K^- \to \mu^- \nu_\mu$
        \item $B^0 \to \bar B^0$
        \item $\tau^- \to \mu^- \mu^+ \mu^-$
    \end{enumerate}

    \subsection{(d)}
    Gibt es im Falle von zwei bzw. drei Quark-Familien CP-Verletzung?
    Begründen Sie kurz.

    \subsection{(e)}
    Welche Voraussetzungen müssen vorliegen, damit es zu Neutrinooszillation kommen kann?

    \subsection{(f)}
    Welche Rolle spielen Dipol- und Quadrupol-Magnete in einem modernen Teilchenbeschleuniger?

    \subsection{(g)}
    Erläutern Sie die Vorteile und Nachteile von Proton-Proton-Kreisbeschleunigern beim Erreichen hoher Schwerpunktsenergien
    im Vergleich zu linearen $e^+ e^-$-Beschleunigern und $e^+ e^-$-Kreisbeschleunigern.
\end{aufgabe}

\begin{aufgabe}{Aufgabe 2: Kernphysik}
    \subsection{(a)}
    Unter welchen Umständen kann ein Kern über die Aussendung eines Positrons und eines Elektronneutrinos zerfallen?

    \subsection{(b)}
    In der Bethe-Weizsäcker-Formel treten ein Oberflächenterm und ein Volumenterm auf.
    Der Oberflächenterm ist proportional zu $A^{2/3}$ und der Volumenterm proportional zu $A$.
    Erklären Sie diese Abhängigkeiten im Tröpfchenmodell.
    Erklären Sie außerdem das relative Vorzeichen der beiden Terme.

    \subsection{(c)}
    \ce{^22_10Ne} und \ce{^22_12Mg} sind Spiegelkerne.
    Was sind Spiegelkerne?
    Welcher Term der Bethe-Weizsäcker-Formel kann mit Spiegelkernen untersucht werden?
    Erklären Sie mithilfe eines Kernmodells,
    warum \ce{^22_10Ne} und \ce{^22_12Mg} sehr ähnliche Anregungsspektren besitzen.

    \subsection{(d)}
    \ce{^211_84Po} und \ce{^207_81Tl} zerfallen in \ce{^207_82Pb} mit den Zerfallskonstanten $\lambda_{\ce{Po}}$ und $\lambda_{\ce{Tl}}$.
    Um welche Arten von Zerfällen handelt es sich?

    \subsection{(e)}
    \ce{^207_82Pb} ist stabil.
    Zum  Zeitpunkt $t = 0$ ist die Besetzungszahl
    von Thallium $N_{\ce{Tl}}(0) = N_0$,
    von Polonium $N_{\ce{Po}}(0) = N_1$ und
    von Blei     $N_{\ce{Pb}}(0) = 0$.
    Stellen Sie die Differenzialgleichungen der drei Isotope auf und berechnen Sie,
    durch das Lösen der Differenzialgleichungen,
    deren Besetzungszahl als Funktion der Zeit.

    \subsection{(f)}
    Nehmen Sie $N_{\ce{Tl}}(0) = N_{\ce{Po}}(0)$ und $\lambda_{\ce{Po}} = 10 \lambda_{\ce{Tl}}$ an.
    Skizzieren Sie qualitativ die Besetzungszahlen der drei Isotope als Funktion der Zeit.
\end{aufgabe}
\end{document}
