\documentclass[
  bibliography=totoc,     % Literatur im Inhaltsverzeichnis
  captions=tableheading,  % Tabellenüberschriften
  titlepage=firstiscover, % Titelseite ist Deckblatt
]{scrartcl}

% Paket float verbessern
\usepackage{scrhack}

% Warnung, falls nochmal kompiliert werden muss
\usepackage[aux]{rerunfilecheck}

% unverzichtbare Mathe-Befehle
\usepackage{amsmath}
% viele Mathe-Symbole
\usepackage{amssymb}
% Erweiterungen für amsmath
\usepackage{mathtools}

% Fonteinstellungen
\usepackage{fontspec}
% Latin Modern Fonts werden automatisch geladen
% Alternativ zum Beispiel:
%\setromanfont{Libertinus Serif}
%\setsansfont{Libertinus Sans}
%\setmonofont{Libertinus Mono}

% Wenn man andere Schriftarten gesetzt hat,
% sollte man das Seiten-Layout neu berechnen lassen
\recalctypearea{}

% deutsche Spracheinstellungen
\usepackage[ngerman]{babel}


\usepackage[
  math-style=ISO,    % ┐
  bold-style=ISO,    % │
  sans-style=italic, % │ ISO-Standard folgen
  nabla=upright,     % │
  partial=upright,   % ┘
  warnings-off={           % ┐
    mathtools-colon,       % │ unnötige Warnungen ausschalten
    mathtools-overbracket, % │
  },                       % ┘
]{unicode-math}

% traditionelle Fonts für Mathematik
\setmathfont{Latin Modern Math}
% Alternativ zum Beispiel:
%\setmathfont{Libertinus Math}

% \setmathfont{XITS Math}[range={scr, bfscr}]
% \setmathfont{XITS Math}[range={cal, bfcal}, StylisticSet=1]

% Zahlen und Einheiten
\usepackage[
  locale=DE,                   % deutsche Einstellungen
  separate-uncertainty=true,   % immer Unsicherheit mit \pm
  per-mode=symbol-or-fraction, % / in inline math, fraction in display math
]{siunitx}

% chemische Formeln
\usepackage[
  version=4,
  math-greek=default, % ┐ mit unicode-math zusammenarbeiten
  text-greek=default, % ┘
]{mhchem}

% richtige Anführungszeichen
\usepackage[autostyle]{csquotes}

% schöne Brüche im Text
\usepackage{xfrac}

% Standardplatzierung für Floats einstellen
\usepackage{float}
\floatplacement{figure}{htbp}
\floatplacement{table}{htbp}

% Floats innerhalb einer Section halten
\usepackage[
  section, % Floats innerhalb der Section halten
  below,   % unterhalb der Section aber auf der selben Seite ist ok
]{placeins}

% Seite drehen für breite Tabellen: landscape Umgebung
\usepackage{pdflscape}

% Captions schöner machen.
\usepackage[
  labelfont=bf,        % Tabelle x: Abbildung y: ist jetzt fett
  font=small,          % Schrift etwas kleiner als Dokument
  width=0.9\textwidth, % maximale Breite einer Caption schmaler
]{caption}
% subfigure, subtable, subref
\usepackage{subcaption}

% Grafiken können eingebunden werden
\usepackage{graphicx}

% schöne Tabellen
\usepackage{booktabs}

% Verbesserungen am Schriftbild
\usepackage{microtype}

% Literaturverzeichnis
\usepackage[
  backend=biber,
]{biblatex}
% Quellendatenbank
\addbibresource{lit.bib}
% \addbibresource{programme.bib}

% Hyperlinks im Dokument
\usepackage[
  german,
  unicode,        % Unicode in PDF-Attributen erlauben
  pdfusetitle,    % Titel, Autoren und Datum als PDF-Attribute
  pdfcreator={},  % ┐ PDF-Attribute säubern
  pdfproducer={}, % ┘
]{hyperref}
% erweiterte Bookmarks im PDF
\usepackage{bookmark}

% Trennung von Wörtern mit Strichen
\usepackage[shortcuts]{extdash}

\usepackage{parskip}

\author{%
  Nicolai Weitkemper\\%
  \href{mailto:nicolai.weitkemper@tu-dortmund.de}{nicolai.weitkemper@tu-dortmund.de}%
}
\publishers{TU Dortmund – Fakultät Physik}

\global\def\colvec#1{\begin{pmatrix}#1\end{pmatrix}}
% \global\def\b#1{\left(#1\right)}
% \global\def\pdv#1{\frac{\partial}{\partial #1}}
\global\def\dv#1#2{\frac{\d #1}{\d #2}}
\global\def\pdv#1#2{\frac{\partial #1}{\partial #2}}
\global\def\pdvfest#1#2#3{\left.\frac{\partial #1}{\partial #2}\right\rvert_{#3}}
\global\def\d{\mathrm{d}}
\global\def\intfty{\int_{-\infty}^\infty} % ha ha
\global\def\half{\frac{1}{2}}
\global\def\quarter{\frac{1}{4}}

\global\def\b#1{\left(#1\right)}
\global\def\L{\mathcal{L}}
\global\def\normtwo{\frac{1}{\sqrt{2}}}


\global\def\kB{k_\text{B}}
\global\def\b#1{\left(#1\right)}
\global\def\a#1{\left\langle#1\right\rangle}

% https://tex.stackexchange.com/a/444226
\makeatletter
\renewcommand{\@seccntformat}[1]{}
\makeatother

\NewCommandCopy{\oldDelta}{\Delta}
\renewcommand{\Delta}{\mathrm{\oldDelta}}

\newenvironment{aufgabe}[1]
    {
    \section{#1}
    }
    {
    \clearpage
    }

% TODO: Nicht die neueste Version des Standard-header_common!


\subject{KET}
\title{Kern- und Elementarteilchenphysik}
\date{
    Wintersemester 2020/2021
}

\begin{document}

\maketitle
\thispagestyle{empty}
\newpage


\begin{aufgabe}{Aufgabe 1: Kurzfragen}
    \subsection{(a)}
    Die Existenz des $\Omega^-$ (Strangeness $S = -3$) lässt sich nur mithilfe eines Farbfreiheitsgrades erklären.
    Warum?

    \subsection{(b)}
    Welche Fusionsreaktion wird für zukünftige Fusionsreaktoren favorisiert?
    Begründen Sie Ihre Entscheidung.
    \subsection{(c)}
    Geben Sie für die folgenden Zerfälle und Reaktionen an, ob sie verboten oder erlaubt sind.
    Begründen Sie Ihre Antworten!
    Skizzieren Sie für die erlaubten Zerfälle jeweils ein Feynman-Diagramm auf Quarkniveau.
    \begin{enumerate}
        \item $K^- \to \mu^- \nu_\mu$
        \item $B^0 \to \bar B^0$
        \item $\tau^- \to \mu^- \mu^+ \mu^-$
    \end{enumerate}

    \subsection{(d)}
    Gibt es im Falle von zwei bzw. drei Quark-Familien CP-Verletzung?
    Begründen Sie kurz.

    \subsection{(e)}
    Welche Voraussetzungen müssen vorliegen, damit es zu Neutrinooszillation kommen kann?

    \subsection{(f)}
    Welche Rolle spielen Dipol- und Quadrupol-Magnete in einem modernen Teilchenbeschleuniger?

    \subsection{(g)}
    Erläutern Sie die Vorteile und Nachteile von Proton-Proton-Kreisbeschleunigern beim Erreichen hoher Schwerpunktsenergien
    im Vergleich zu linearen $e^+ e^-$-Beschleunigern und $e^+ e^-$-Kreisbeschleunigern.
\end{aufgabe}

\begin{aufgabe}{Aufgabe 2: Kernphysik}
    \subsection{(a)}
    Unter welchen Umständen kann ein Kern über die Aussendung eines Positrons und eines Elektronneutrinos zerfallen?

    \subsection{(b)}
    In der Bethe-Weizsäcker-Formel treten ein Oberflächenterm und ein Volumenterm auf.
    Der Oberflächenterm ist proportional zu $A^{2/3}$ und der Volumenterm proportional zu $A$.
    Erklären Sie diese Abhängigkeiten im Tröpfchenmodell.
    Erklären Sie außerdem das relative Vorzeichen der beiden Terme.

    \subsection{(c)}
    \ce{^22_10Ne} und \ce{^22_12Mg} sind Spiegelkerne.
    Was sind Spiegelkerne?
    Welcher Term der Bethe-Weizsäcker-Formel kann mit Spiegelkernen untersucht werden?
    Erklären Sie mithilfe eines Kernmodells,
    warum \ce{^22_10Ne} und \ce{^22_12Mg} sehr ähnliche Anregungsspektren besitzen.

    \subsection{(d)}
    \ce{^211_84Po} und \ce{^207_81Tl} zerfallen in \ce{^207_82Pb} mit den Zerfallskonstanten $\lambda_{\ce{Po}}$ und $\lambda_{\ce{Tl}}$.
    Um welche Arten von Zerfällen handelt es sich?

    \subsection{(e)}
    \ce{^207_82Pb} ist stabil.
    Zum  Zeitpunkt $t = 0$ ist die Besetzungszahl
    von Thallium $N_{\ce{Tl}}(0) = N_0$,
    von Polonium $N_{\ce{Po}}(0) = N_1$ und
    von Blei     $N_{\ce{Pb}}(0) = 0$.
    Stellen Sie die Differenzialgleichungen der drei Isotope auf und berechnen Sie,
    durch das Lösen der Differenzialgleichungen,
    deren Besetzungszahl als Funktion der Zeit.

    \subsection{(f)}
    Nehmen Sie $N_{\ce{Tl}}(0) = N_{\ce{Po}}(0)$ und $\lambda_{\ce{Po}} = 10 \lambda_{\ce{Tl}}$ an.
    Skizzieren Sie qualitativ die Besetzungszahlen der drei Isotope als Funktion der Zeit.
\end{aufgabe}
\end{document}
