\input{header_common.tex}

\subject{KET}
\title{Kern- und Elementarteilchenphysik}
\date{
    Wintersemester 2020/2021
}

\begin{document}

\maketitle
\thispagestyle{empty}
\newpage


\begin{aufgabe}{Aufgabe 1: Kurzfragen}
\subsection{(a)}
Was ist die Bedeutung des Formfaktors in Streureaktionen der Kern- und Teilchenphysik,
z.B. bei der Streuung von Elektronen an Kernen?

\subsection{(b)}
Die Suche nach neuen Teilchen ist eine zentrale Aufgabe der Experimente ATLAS und CMS.
Ergibt eine Suche keinen Erfolg, wird eine Massenobergrenze für die hypothetischen Teilchen bestimmt.
Sie stellen fest, dass diese Grenzen in der Regel unterhalb von $\SI{3}{\tera\electronvolt}/c^2$ liegen,
obwohl die Schwerpunktsenergie \SI{13}{\tera\electronvolt} betragen hat.
Wie kann dieser Unterschied erklärt werden?

\subsection{(c)}
Warum müssen Kernreaktoren auch nach deren Abschaltung gekühlt werden?

\subsection{(d)}
Was bedeuten die Begriffe \enquote{asymptotic freedom} und \enquote{confinement}?

\subsection{(e)}
Sind Flavour-Changing-Neutral-Currents (FCNC) im Standardmodell auf Tree-Niveau erlaubt?
Begründen Sie Ihre Antwort.

\subsection{(f)}
Was sagt Fermis goldene Regel aus?

\subsection{(g)}
Schätzen Sie die relativen Zerfallsraten der folgenden Higgs-Zerfälle quantitativ ab:
$H \to e^+ e^−$,
$H \to \mu^+ \mu^-$,
$H \to \tau^+ \tau^−$.

\subsection{(h)}
Geben Sie für die folgenden Zerfälle und Reaktionen an, ob sie verboten oder erlaubt
sind. Begründen Sie Ihre Antworten! Skizzieren Sie für die erlaubten Zerfälle jeweils ein
Feynman-Diagramm führender Ordnung auf Quarkniveau.
\begin{enumerate}
    \item $B_s^0 \to \mu^+ \mu^-$
    \item $p \to \pi^+ e^+ e^−$
    \item $D_s^+ \to \tau^+ \nu_\tau$
\end{enumerate}
\end{aufgabe}
\end{document}
