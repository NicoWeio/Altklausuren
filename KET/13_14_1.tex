\documentclass[
  bibliography=totoc,     % Literatur im Inhaltsverzeichnis
  captions=tableheading,  % Tabellenüberschriften
  titlepage=firstiscover, % Titelseite ist Deckblatt
]{scrartcl}

% Paket float verbessern
\usepackage{scrhack}

% Warnung, falls nochmal kompiliert werden muss
\usepackage[aux]{rerunfilecheck}

% unverzichtbare Mathe-Befehle
\usepackage{amsmath}
% viele Mathe-Symbole
\usepackage{amssymb}
% Erweiterungen für amsmath
\usepackage{mathtools}

% Fonteinstellungen
\usepackage{fontspec}
% Latin Modern Fonts werden automatisch geladen
% Alternativ zum Beispiel:
%\setromanfont{Libertinus Serif}
%\setsansfont{Libertinus Sans}
%\setmonofont{Libertinus Mono}

% Wenn man andere Schriftarten gesetzt hat,
% sollte man das Seiten-Layout neu berechnen lassen
\recalctypearea{}

% deutsche Spracheinstellungen
\usepackage[ngerman]{babel}


\usepackage[
  math-style=ISO,    % ┐
  bold-style=ISO,    % │
  sans-style=italic, % │ ISO-Standard folgen
  nabla=upright,     % │
  partial=upright,   % ┘
  warnings-off={           % ┐
    mathtools-colon,       % │ unnötige Warnungen ausschalten
    mathtools-overbracket, % │
  },                       % ┘
]{unicode-math}

% traditionelle Fonts für Mathematik
\setmathfont{Latin Modern Math}
% Alternativ zum Beispiel:
%\setmathfont{Libertinus Math}

% \setmathfont{XITS Math}[range={scr, bfscr}]
% \setmathfont{XITS Math}[range={cal, bfcal}, StylisticSet=1]

% Zahlen und Einheiten
\usepackage[
  locale=DE,                   % deutsche Einstellungen
  separate-uncertainty=true,   % immer Unsicherheit mit \pm
  per-mode=symbol-or-fraction, % / in inline math, fraction in display math
]{siunitx}

% chemische Formeln
\usepackage[
  version=4,
  math-greek=default, % ┐ mit unicode-math zusammenarbeiten
  text-greek=default, % ┘
]{mhchem}

% richtige Anführungszeichen
\usepackage[autostyle]{csquotes}

% schöne Brüche im Text
\usepackage{xfrac}

% Standardplatzierung für Floats einstellen
\usepackage{float}
\floatplacement{figure}{htbp}
\floatplacement{table}{htbp}

% Floats innerhalb einer Section halten
\usepackage[
  section, % Floats innerhalb der Section halten
  below,   % unterhalb der Section aber auf der selben Seite ist ok
]{placeins}

% Seite drehen für breite Tabellen: landscape Umgebung
\usepackage{pdflscape}

% Captions schöner machen.
\usepackage[
  labelfont=bf,        % Tabelle x: Abbildung y: ist jetzt fett
  font=small,          % Schrift etwas kleiner als Dokument
  width=0.9\textwidth, % maximale Breite einer Caption schmaler
]{caption}
% subfigure, subtable, subref
\usepackage{subcaption}

% Grafiken können eingebunden werden
\usepackage{graphicx}

% schöne Tabellen
\usepackage{booktabs}

% Verbesserungen am Schriftbild
\usepackage{microtype}

% Literaturverzeichnis
\usepackage[
  backend=biber,
]{biblatex}
% Quellendatenbank
\addbibresource{lit.bib}
% \addbibresource{programme.bib}

% Hyperlinks im Dokument
\usepackage[
  german,
  unicode,        % Unicode in PDF-Attributen erlauben
  pdfusetitle,    % Titel, Autoren und Datum als PDF-Attribute
  pdfcreator={},  % ┐ PDF-Attribute säubern
  pdfproducer={}, % ┘
]{hyperref}
% erweiterte Bookmarks im PDF
\usepackage{bookmark}

% Trennung von Wörtern mit Strichen
\usepackage[shortcuts]{extdash}

\usepackage{parskip}

\author{%
  Nicolai Weitkemper\\%
  \href{mailto:nicolai.weitkemper@tu-dortmund.de}{nicolai.weitkemper@tu-dortmund.de}%
}
\publishers{TU Dortmund – Fakultät Physik}

\global\def\colvec#1{\begin{pmatrix}#1\end{pmatrix}}
% \global\def\b#1{\left(#1\right)}
% \global\def\pdv#1{\frac{\partial}{\partial #1}}
\global\def\dv#1#2{\frac{\d #1}{\d #2}}
\global\def\pdv#1#2{\frac{\partial #1}{\partial #2}}
\global\def\pdvfest#1#2#3{\left.\frac{\partial #1}{\partial #2}\right\rvert_{#3}}
\global\def\d{\mathrm{d}}
\global\def\intfty{\int_{-\infty}^\infty} % ha ha
\global\def\half{\frac{1}{2}}
\global\def\quarter{\frac{1}{4}}

\global\def\b#1{\left(#1\right)}
\global\def\L{\mathcal{L}}
\global\def\normtwo{\frac{1}{\sqrt{2}}}


\global\def\kB{k_\text{B}}
\global\def\b#1{\left(#1\right)}
\global\def\a#1{\left\langle#1\right\rangle}

% https://tex.stackexchange.com/a/444226
\makeatletter
\renewcommand{\@seccntformat}[1]{}
\makeatother

\NewCommandCopy{\oldDelta}{\Delta}
\renewcommand{\Delta}{\mathrm{\oldDelta}}

\newenvironment{aufgabe}[1]
    {
    \section{#1}
    }
    {
    \clearpage
    }

% TODO: Nicht die neueste Version des Standard-header_common!


\subject{KET}
\title{Kern- und Elementarteilchenphysik}
\date{
    Wintersemester 2013/2014
}

\begin{document}

\maketitle
\thispagestyle{empty}
\newpage


\begin{aufgabe}{Aufgabe 1: Kurzfragen}
    Beantworten Sie die nachfolgenden Fragen mit einer kurzen Antwort.
    Gegebenenfalls darf es auch eine Skizze, eine Formel oder ein Zahlenwert sein.
    Denken Sie auch an korrekte Beschriftungen und Einheiten.
    Aber umfangreiche Rechnungen sind hier nicht gefordert, genauso wenig wie lange Antwortsätze.

    \subsection{a)}
    Wie groß ist in etwa die Ruhemasse des Higgs-Bosons?

    \subsection{b)}
    Wie groß ist in etwa die beobachtete Lebensdauer eines freien Myons,
    das sich mit einer Energie von \SI{106}{\giga\electronvolt} im Laborsystem bewegt?

    \subsection{c)}
    Gegeben ist das folgende $J^P = 1/2^+$ Baryon-Oktett.
    Ergänzen Sie die fehlenden Quarkinhalte sowie die Achsenbeschriftungen.

    \subsection{d)}
    Schätzen Sie das Zerfallsbreitenverhältnis des $b$-Quarks ab.
    Vernachlässigen Sie den Einfluss von Massenunterschieden zwischen den Zerfallsprodukten.

    \subsubsection{(i)}
    \[
        \frac{\Gamma(b \to c \mu^-\bar\nu_\mu)}{{\Gamma(b \to u \mu^-\bar\nu_\mu)}}
    \]

    \subsubsection{(ii)}
    \[
        \frac{\Gamma(b \to cs\bar u)}{{\Gamma(b \to u \tau^-\bar\nu_\tau)}}
    \]

    \subsection{e)}
    Was versteht man unter einem Parton?
    Geben Sie ein Beispiel.

    \subsection{f)}
    Welche Quantenzahlen werden von der schwachen Wechselwirkung \underline{nicht} erhalten?
    Zwei Beispiele genügen.

    \subsection{g)}
    Beziffern Sie ungefähr die Masse des Antiteilchens zum Positron.

    \subsection{h)}
    Was versteht man unter Neutrino-Oszillationen?

    \subsection{i)}
    Quantifizieren Sie die natürliche Strahlenbelastung in unserer Umwelt.

    \subsection{j)}
    Wie kann man sich vor Alpha-Strahlung schützen?

    \subsection{k)}
    Warum binden zwei Neutronen nicht zu einem Di-Neutron?

    \subsection{l)}
    Welche Aufgabe hat ein Moderator in der Kernphysik?

    \subsection{m)}
    Was versteht man unter der Bethe-Weizsäcker-Massenformel?

    \subsection{n)}
    Skizzieren Sie graphisch den Energieverlust schwerer, geladener Teilchen in Materie.
    Denken Sie an eine sinnvolle Achsenbeschriftung.

    \subsection{o)}
    Welchen Radius hat etwa ein \ce{Pb}-208-Nuklid?

    \subsection{p)}
    Wie groß ist in etwa die Bindungsenergie der Nukleonen in einem Deuteron?

    \subsection{q)}
    Welche Faktoren bestimmen das Verhältnis $R$ der Wirkungsquerschnitte
    $R = \sigma(e^+e^- \to q\bar q) / \sigma(e^+e^- \to \mu\bar \mu)$?

    \subsection{r)}
    Was versteht man unter einer CPT-Transformation?
    Und wie verhalten sich physikalische Systeme unter dieser Transformation?

    \subsection{s)}
    Skizzieren Sie schematisch den Aufbau eines großen Detektors an einem Collider
    und benennen Sie die wichtigsten Komponenten.

    \subsection{t)}
    Nennen Sie teilchenphysikalische Aspekte,
    den das Standardmodell der Teilchenphysik nicht beschreibt.
    Zwei Aspekte mögen genügen.
\end{aufgabe}
\end{document}
