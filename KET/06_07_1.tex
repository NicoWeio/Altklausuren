\documentclass[
  bibliography=totoc,     % Literatur im Inhaltsverzeichnis
  captions=tableheading,  % Tabellenüberschriften
  titlepage=firstiscover, % Titelseite ist Deckblatt
]{scrartcl}

% Paket float verbessern
\usepackage{scrhack}

% Warnung, falls nochmal kompiliert werden muss
\usepackage[aux]{rerunfilecheck}

% unverzichtbare Mathe-Befehle
\usepackage{amsmath}
% viele Mathe-Symbole
\usepackage{amssymb}
% Erweiterungen für amsmath
\usepackage{mathtools}

% Fonteinstellungen
\usepackage{fontspec}
% Latin Modern Fonts werden automatisch geladen
% Alternativ zum Beispiel:
%\setromanfont{Libertinus Serif}
%\setsansfont{Libertinus Sans}
%\setmonofont{Libertinus Mono}

% Wenn man andere Schriftarten gesetzt hat,
% sollte man das Seiten-Layout neu berechnen lassen
\recalctypearea{}

% deutsche Spracheinstellungen
\usepackage[ngerman]{babel}


\usepackage[
  math-style=ISO,    % ┐
  bold-style=ISO,    % │
  sans-style=italic, % │ ISO-Standard folgen
  nabla=upright,     % │
  partial=upright,   % ┘
  warnings-off={           % ┐
    mathtools-colon,       % │ unnötige Warnungen ausschalten
    mathtools-overbracket, % │
  },                       % ┘
]{unicode-math}

% traditionelle Fonts für Mathematik
\setmathfont{Latin Modern Math}
% Alternativ zum Beispiel:
%\setmathfont{Libertinus Math}

% \setmathfont{XITS Math}[range={scr, bfscr}]
% \setmathfont{XITS Math}[range={cal, bfcal}, StylisticSet=1]

% Zahlen und Einheiten
\usepackage[
  locale=DE,                   % deutsche Einstellungen
  separate-uncertainty=true,   % immer Unsicherheit mit \pm
  per-mode=symbol-or-fraction, % / in inline math, fraction in display math
]{siunitx}

% chemische Formeln
\usepackage[
  version=4,
  math-greek=default, % ┐ mit unicode-math zusammenarbeiten
  text-greek=default, % ┘
]{mhchem}

% richtige Anführungszeichen
\usepackage[autostyle]{csquotes}

% schöne Brüche im Text
\usepackage{xfrac}

% Standardplatzierung für Floats einstellen
\usepackage{float}
\floatplacement{figure}{htbp}
\floatplacement{table}{htbp}

% Floats innerhalb einer Section halten
\usepackage[
  section, % Floats innerhalb der Section halten
  below,   % unterhalb der Section aber auf der selben Seite ist ok
]{placeins}

% Seite drehen für breite Tabellen: landscape Umgebung
\usepackage{pdflscape}

% Captions schöner machen.
\usepackage[
  labelfont=bf,        % Tabelle x: Abbildung y: ist jetzt fett
  font=small,          % Schrift etwas kleiner als Dokument
  width=0.9\textwidth, % maximale Breite einer Caption schmaler
]{caption}
% subfigure, subtable, subref
\usepackage{subcaption}

% Grafiken können eingebunden werden
\usepackage{graphicx}

% schöne Tabellen
\usepackage{booktabs}

% Verbesserungen am Schriftbild
\usepackage{microtype}

% Literaturverzeichnis
\usepackage[
  backend=biber,
]{biblatex}
% Quellendatenbank
\addbibresource{lit.bib}
% \addbibresource{programme.bib}

% Hyperlinks im Dokument
\usepackage[
  german,
  unicode,        % Unicode in PDF-Attributen erlauben
  pdfusetitle,    % Titel, Autoren und Datum als PDF-Attribute
  pdfcreator={},  % ┐ PDF-Attribute säubern
  pdfproducer={}, % ┘
]{hyperref}
% erweiterte Bookmarks im PDF
\usepackage{bookmark}

% Trennung von Wörtern mit Strichen
\usepackage[shortcuts]{extdash}

\usepackage{parskip}

\author{%
  Nicolai Weitkemper\\%
  \href{mailto:nicolai.weitkemper@tu-dortmund.de}{nicolai.weitkemper@tu-dortmund.de}%
}
\publishers{TU Dortmund – Fakultät Physik}

\global\def\colvec#1{\begin{pmatrix}#1\end{pmatrix}}
% \global\def\b#1{\left(#1\right)}
% \global\def\pdv#1{\frac{\partial}{\partial #1}}
\global\def\dv#1#2{\frac{\d #1}{\d #2}}
\global\def\pdv#1#2{\frac{\partial #1}{\partial #2}}
\global\def\pdvfest#1#2#3{\left.\frac{\partial #1}{\partial #2}\right\rvert_{#3}}
\global\def\d{\mathrm{d}}
\global\def\intfty{\int_{-\infty}^\infty} % ha ha
\global\def\half{\frac{1}{2}}
\global\def\quarter{\frac{1}{4}}

\global\def\b#1{\left(#1\right)}
\global\def\L{\mathcal{L}}
\global\def\normtwo{\frac{1}{\sqrt{2}}}


\global\def\kB{k_\text{B}}
\global\def\b#1{\left(#1\right)}
\global\def\a#1{\left\langle#1\right\rangle}

% https://tex.stackexchange.com/a/444226
\makeatletter
\renewcommand{\@seccntformat}[1]{}
\makeatother

\NewCommandCopy{\oldDelta}{\Delta}
\renewcommand{\Delta}{\mathrm{\oldDelta}}

\newenvironment{aufgabe}[1]
    {
    \section{#1}
    }
    {
    \clearpage
    }

% TODO: Nicht die neueste Version des Standard-header_common!


\subject{KET}
\title{Kern- und Elementarteilchenphysik}
\date{
    Wintersemester 2006/2007
}

\begin{document}

\maketitle
\thispagestyle{empty}
\newpage


\begin{aufgabe}{Aufgabe 1: Kurzfragen}
    \subsection{1.}
    Worin liegt eigentlich der Unterschied zwischen Kern- und Teilchenphysik?

    \subsection{2.}
    Wie groß ist ungefähr der Fermi-Impuls der Nukleonen in mittelschweren Kernen?

    \subsection{3.}
    Wie kann man verstehen, dass das magnetische Moment des Neutrons von Null verschieden ist?

    \subsection{4.}
    Mit welcher experimentellen und mathematischen Methode kann man das Ladungsdichteprofil von Kernen ermitteln?

    \subsection{5.}
    Skizzieren Sie den ungefähren radialen Verlauf der Ladungsdichte eines leichten Kerns wie \ce{^6Li}.
    Kennzeichen Sie die Achsen Ihrer Darstellung.

    \subsection{6.}
    Was ist der Unterschied zwischen Gamma-Strahlung und Röntgenstrahlung?
    (2 Charakteristika genügen)

    \subsection{7.}
    Welchen Radius hat ungefähr ein Blei-Kern?
    Wie groß ist ungefähr ein Proton?
    Und wie groß ist ungefähr ein Elektron?
    Wie groß ist ein Wasserstoff-Atom?

    \subsection{8.}
    Sie möchten experimentelle Aussagen über die Größe-von Gold-Kernen in Streuexperimenten machen.
    Welchen Teilchenstrahl könnte man verwenden?
    Bei welchen Impulsen erwarten Sie eine Punktförmigkeit des Kernes,
    bei welchen Impulsen lässt sich die endliche Ausdehnung erkennen?

    \subsection{9.}
    Was versteht man unter Rutherford-Streuung?

    \subsection{10.}
    Auf welche experimentellen Befunde konnte Pauli seine Neutrino-Hypothese stützen?

    \subsection{11.}
    Welche Eigenschaften haben Neutrinos?
    (3 Charakteristika genügen)

    \subsection{12.}
    Was versteht man unter einem Formfaktor im Rahmen der Kernphysik?

    \subsection{13.}
    Welche wesentlichen Faktoren gehen in die Rosenbluth-Formel ein?

    \subsection{14.}
    Geben Sie die Reaktionsgleichung für $\beta^-$-Zerfälle auf Nukleonen- und auf Quarkniveau an.
    Skizzieren Sie den Feynman-Graphen auf Quarkniveau.

    \subsection{15.}
    Was verstehen Sie unter radioaktivem Gleichgewicht?

    \subsection{16.}
    Skizzieren Sie die Massenverteilung der Kernbruchstücke bei der Spaltung von \ce{^235U}.

    \subsection{17.}
    Welche Bedeutung haben die \enquote{verzögerten Neutronen} für die Regelung von Kernspaltungsreaktoren°?

    \subsection{18.}
    Skizzieren Sie die prinzipiellen Verläufe der Spalt-Wirkungsquerschnitte $(n, f)$ für \ce{^238U} und \ce{^235U} als Funktion der Neutronenenergie.

    \subsection{19.}
    Wie groß sind die typischen kinetischen Energien der bei Kernspaltungen freigesetzten Neutronen?

    \subsection{20.}
    Welche Aufgaben hat ein Moderator in einem Spaltreaktor?

    \subsection{21.}
    Mit welchem Moderator und welchem primären Kühlmittel arbeiten Reaktoren des Typs RBMK (Tschernobyl)?
    Welchen Nachteil hat diese Kombination?

    \subsection{22.}
    Wie rechnet man eine \enquote{Energiedosis} in eine \enquote{Äquivalentdosis} um,
    und in welchen Einheiten misst man diese Größen?

    \subsection{23.}
    Wie groß ist etwa die jährliche natürliche Strahlenbelastung in Deutschland
    und was liefert die größten Beiträge hierzu?

    \subsection{24.}
    Welche beiden sehr unterschiedlichen Wege zur Realisierung einer technischen Kernfusion werden verfolgt?

    \subsection{25.}
    Welcher Reaktortyp soll in Cadarache realisiert werden?

    \subsection{26.}
    Was ist der dominante Energieverlustmechanismus für schwere geladene Teilchen?

    \subsection{27.}
    Was ist ein \enquote{mip}?

    \subsection{28.}
    Welche der drei Wechselwirkungsarten von Photonen mit Materie hängen stark von der Kernladungszahl ab?
    Warum?

    \subsection{29.}
    Was sind wesentliche Unterschiede zwischen elektromagnetischen und hadronischen Schauern?

    \subsection{30.}
    Was unterscheidet Driftkammern von Proportionalkammern?

    \subsection{31.}
    Was ist ein \enquote{sampling Kalorimeter}?

    \subsection{32.}
    Skizzieren Sie den schematischen Aufbau eines modernen Detektors an einem Collider mit seinen wesentlichen Detektorelementen
    und benennen Sie diese.

    \subsection{33.}
    Welche Elementarteilchen unterliegen der starken Wechselwirkung?
    Geben Sie drei Beispiele!

    \subsection{34.}
    Welche Elementarteilchen unterliegen der schwachen Wechselwirkung?
    Geben Sie drei Beispiele!

    \subsection{35.}
    Welche Elementarteilchen unterliegen \underline{nicht} der Gravitation?

    \subsection{36.}
    Welche ungefähre Masse haben Myonen und Pionen?

    \subsection{37.}
    Skizzieren Sie das Multiplett der leichtesten Baryonen ($\operatorname{SU}(3)$)!
    Beschriften Sie dabei auch die Achsen Ihrer Darstellung!

    \subsection{38.}
    Welcher Spinstatistik unterliegen Baryonen?

    \subsection{39.}
    Sie entdecken beim Experimentieren einen $\bar u \bar d \bar s$-Zustand.
    Wie sollten Sie den Zustand möglicherweise nennen?

    \subsection{40.}
    Was versteht man unter Paritätsverletzung?
    Geben Sie ein Beispiel!

    \subsection{41.}
    Was versteht man unter CP-Verletzung?
    Geben Sie ein Beispiel!

    \subsection{42.}
    Welches ist das massenreichste bisher bekannte Elementarteilchen?

    \subsection{43.}
    Wie viele Zustände kann man auf ein Dekuplett verteilen?

    \subsection{44.}
    Sie entdecken nach ausgiebigem Experimentieren mit einem respektablen Konfidenzniveau den Zerfall $p \to e^+ \gamma$.
    Welche Kapitel für eine Neuauflage eines herauszubringenden Lehrbuchs über Teilchenphysik müssten bei Bestätigung Ihrer Resultate neu geschrieben werden?
    Geben Sie dazu einige Stichworte.

    \subsection{45.}
    Sie beobachten eine Reaktion $e^+ e^- \to \tau^+ \tau^-$.
    Welche Wechselwirkung liegt dieser Reaktion zugrunde?
    Nach weiteren Analysen entdecken Sie auch den Kanal $e^+ e^- \to \nu_\tau \bar\nu_\tau$.
    Welche Wechselwirkung vermuten Sie hier?

    \subsection{46.}
    Wie zerfallen geladene Pionen?
    Geben Sie dazu ein Beispiel einer Reaktionsgleichung.
    Veranschaulichen Sie den Zerfall mithilfe eines Feynman-Graphen.

    \subsection{47.}
    In welcher Wechselwirkung können Kaonen produziert werden?
    Unter welcher Wechselwirkung zerfallen Kaonen?

    \subsection{48.}
    Was versteht man unter der Ladungskonjugation C?

    \subsection{49.}
    Wie lautet das Antiteilchen zum neutralen Pion?
    Wie lautet das Antiteilchen zum Neutron?
\end{aufgabe}
\end{document}
